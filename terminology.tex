
\label{sec:terminology}
Implementierungstechnisch schränkt eine Rechenmaschine die mathematischen Zahlenräume ein:
\begin{itemize}
\item Die Reelen Zahlen $\mathbb{R}$ beschränken sich auf Floating-Point-Datentypen, welche hier im weiteren mit $\mathcal{F} \subset \mathbb{R}$ bezeichnet werden
\item Integern $\mathbb{Z}$ sind maschinell in ihrer Darstellungsgröße beschränkt. Diese beschränkte Menge an Integern wird $\mathcal{I} \subset \mathbb{Z}$ genannt
\end{itemize}


\subsection{Zeit}
\label{sec:time}
\def\finite#1{\ooalign{\hfil$\mapstochar\mkern 3mu\mapstochar\mkern 5mu$\hfil\cr$#1$}}

Jede Simulationsinstanz, d.h. eine Simulation auf einer konkreten Maschine, läuft gezeitet ab. Es existieren je zwei relevante zeitliche Sequenzen:
\begin{enumerate}
\item Die durch maschinelle Abtastung diskrete Realzeit der echten Welt\\
	\begin{itemize}
	\item $T_r:=\langle t_{epoch}, ... , t_{max}\rangle ; t_{epoch}, t_{max}$ als minimal, bzw.~maximal darstellbare Zeit
	\item die in einer maschinellen Genauigkeit in Mikrosekunden $$\epsilon_t:=10^{-6}s ; \forall c \in \mathbb{Z}: T_r(c) + \epsilon_t = T_r(c+1)$$ gegeben ist
	\item und immer monoton wächst $\forall c \in \mathbb{Z}: T_r(c) < T_r(c+1)$. 
	\end{itemize}
	

\item Die Simulationszeit $T_s:=\langle t_{start}, ... , t_{end}\rangle; t_{start}, t_{end}$ als Start- und Endzeitpunkt der Simulation.
	\begin{itemize}
	\item Zwischen den beiden Zeitbasen besteht eine totale, nicht-injektive, surjektive Abbildung $\mathcal{T}:T_r \twoheadrightarrow T_s$
	\item Die Simulationszeit ist dadurch relativ zur Realzeit definiert $T_s:=\langle\mathcal{T}\rangle$
	\item Um die Kontinuität der Zeit herzustellen wird weiter eine Zeittate $r_t:T_r\mapsto\mathcal{F}$ definiert, welche das relative verstreichen der Zeit in der Simulation steuert. 
	$$\forall t_{r0},  t_{r1} \in T_r ; t_{diff}=t_{r1}-t_{r0} :\mathcal{T}(t_{r1}) = \mathcal{T}(t_{r0}) + t_{diff}*r_t(t_{r1})$$ unter der Vorraussetzung, dass die Rate während aller Zeiten zwischen $t_{r0}$ und $t_{r1}$ gleich bleibt $\forall t \in [ t_{r0},t_{r1}]: r_t( t_{r0}) = r_t(t)$. Wird die Eigenschaft des Zeitflusses in der festgelegten Rate verletzt, werden die aktuellen Echtzeitanforderungen verletzt. \\
	Soll die Rate also geändert werden muss dies zu festgelegten Umschaltpunkten geschehen, welche die Simulationszeit in Zeitbereiche trennen, zwischen denen keine Berechnungsvorgänge zuverlässig durchführbar sind.\\
Beispiele für Raten sind 
\begin{itemize}
\item $r_t(x) = 1 \Leftrightarrow$ Simulation synchron zur Echtzeit
\item $r_t(x) = 0 \Leftrightarrow$ Simulation ist pausiert/läuft nicht
\end{itemize}
Technisch wird für große $|r_t|$ die Simulation schwierig, da viele Vorgänge schnell simuliert werden müssten. Diese werden daher vermieden.\\
Theoretisch kann die Rate auch negative Werte annehmen. Die Simulationszeit würde dann rückwärts laufen. Dieses Verhalten ist technisch durch die monoton steigende Realzeit nicht leicht in konsistenter Weise umzusetzen, da Realzeiteinflüsse durch Tasteneingaben existieren und soll daher hier ebenfalls vermieden werden.
	\end{itemize}
\end{enumerate}	


\subsection{Tick \& Frame}

Die Simulation behandelt das Verstreichen von Zeit in Zeitschritten, während dem der interne Zustand der Simulation, bzw.~der simulierten Objekte, zu einem zeitlich neuen Zustand aktualisiert wird.
Dieser Zeitschritt, bzw.~Verarbeitungsschritt, wird oft als Tick bezeichnet.\\

Es ist besonders anzumerken, das der Begriff des Ticks sich ausschließlich auf das Voranschreiten der Simulation bezieht und nicht dem Anzeigen einer Szene. Die Äquivalente Bezeichnung im Kontext der Grafik wird als Frame bezeichnet, in welchem eine Szene (ein Grafischer Zustand der Simulation) gerendert wird. Es besteht Verwechslungsgefahr. Von beiden Größen können Raten $r_{tick}, r_{frame}$ angegeben werden (üblicherweise in Tick/Frame pro Realzeitsekunde). Praktisch kann eine Grafikengine durch Inter- oder Extrapolation unterschiedliche Raten erreichen.\\
Wir definieren die Menge der Ticks $\delta:T_r^2; \delta:=\{\delta_1, \delta_2, ...\}$ anhand ihrer Start- und Endzeitpunkte in Echtzeit $\delta_i := (\delta_{i0}, \delta_{i1})$ und erweitert die zu einem Tick gehörenden Zeitpunkte als $\delta_{id}; d \in [0,1]$. Die Inklusivität/Exklusivität muss in bestimmten Berechnungskontexten manchmal angepasst werden um bei sukzessiven Ticks doppelte Behandlungen von Ereignissen zu vermeiden. Diese Einschätzung sei für jeden Kontext individuell zu vollziehen.\\
Es gilt außerdem die Kontinuität der Zeit auch bei Ticks $\delta_{j1} = \delta_{(j+1)0}$, d.h. ein Tick beginnt am Endzeitpunkt des Vorherigen.\\
Durch die Abbildung $\mathcal{T}$ erhält der Tick eine Entsprechung in Simulationszeit.\\
Ist im aktuellen Kontext nur ein Tick $\delta_i$ von belang wird auch die Terminologie $t_d =\mathcal{T}(\delta_{id})$, also $t_0$ für die Tickbeginn und $t_1$ für das Ende in Simulationszeit, verwendet.\\
Man kann weiter die Sequenz $\Upsilon_{\delta i} = \langle t_0, ...,  t_1\rangle$ als die zusammenhängende Partition der Simulationszeitsequenz $T_s$ denotieren, welche die geordneten Zeitpunkte eines Ticks in Simulationszeit enthält.\\
Die in Abschnitt~\ref{sec:time} beschriebenen möglichen Umschaltzeiten zur Änderung der Zeitflussrate in der Simulation werden auf die Grenzen von Ticks gelegt.\\
Die Größe der Zeitdifferenz $t_1 - t_0$ unterliegt meist Einschränkungen. Bestimmte Simulationsalgorithmen wie z.B. die Methode der kleinen Schritte erfordert für eine bestimmte Genauigkeit eine maximale Schrittgröße. Die verfügbare Rechenleistung hingegen beschränkt die Tickrate nach oben. Reicht die Berechnungszeit während eine Ticks nicht um den Status der Simulation von $t_0$ auf $t_1$ zu aktualisieren, läuft die Simulation langsamer als die reale Zeit. Die Echtzeitanforderung ist dann verletzt. Oft wird die Tickrate als Konstante festgelegt, in diesem Projekt ist jedoch nur eine Mindestrate festgelegt.


\subsection{Raum}
\label{sec:space}
Der geforderte 3D Raum kann durch 3-dimensionale Vektoren $\in \mathcal{F}^3$ in der Einheit Meter beschrieben werden.\\
Durch die Werteverteilung in $\mathcal{F}$ treten jedoch bei großen Räumen für Positionen mit großer Entfernung zum Ursprung $O$ Genauigkeitsdefizite auf, die zur Verletzung von Genauigkeitsanforderung führen können. Mögliche Floating-Point-Werte liegen dabei dichter beieinander, je näher am Ursprung \cite{floatdistribution}. Ein Beispiel für die Auswirkungen dieses Sachverhalts in Simulationen kann in der Quelle \cite{floatdistributionexample} betrachtet werden.\\
Physikalische Prozesse berechnet auf Basis von Positionen in $\mathcal{F}^3$ können daher inkonsistent in Abhängigkeit zum Ort im Raum sein.\\
Das Problem wird hier durch einen neuen Längendatentypen $\mathcal{S} : \mathcal{I} \times \mathcal{F}$ gelöst, welcher den Raum zunächst gleichmäßig durch $\mathcal{I}$ aufteilt und indiziert und $\mathcal{F}$ als Offset innerhalb seines Raumteils verwendet. Es wird daher eine Größe der initialen Aufteilung $size_{grid}$ definiert.\\
Die Umrechnung zu Metern ist dann: $$ meter: \mathcal{S} \mapsto \mathcal{F};  meter((i, f)) = i * size_{grid} + f * size_{grid}$$ 
Typischerweise gilt $f \in [0;1[$, um eine eindeutige Repräsentation für einen beschriebenen Punkt zu erhalten.

Diese Darstellung hat folgende weitere Vorteile
\begin{itemize}
\item Einfache Implementierung
\item Schnelle Indizierung der durch $\mathcal{I}$ indizierten Raumanteile für raumaufteilende Teile-und-Herrsche-Algorithmen
\end{itemize}

Absolute Positionen im Raum werden demnach mit Vektoren $s\in\mathcal{S}^3$ dargestellt. Für Berechnungen von Interaktionen zwischen Objekten werden Positionen zunächst relativiert, d.h. Positionen $p \in \mathcal{S}$ sollen zu $p_0$ relativ gesetzt werden, dann sind die relativen Positionen $p' = p - p_0$. Diese werden dann in in die für lokale Interaktionen sinnhafte  Einheit Meter $\mathcal{F}^3$ umgewandelt, um darauf Berechnungen durchzuführen. Man geht dabei davon aus, das relative Strecken zwischen Objekten kurz genug sind, sodass die Genauigkeitsänderung in $\mathcal{F}$ vernachlässigbar ist.\\
Effektiv ist dabei die Eigenschaft $\mathcal{F}\subset\mathcal{S}$ nicht gefordert, auch wenn sie in der in diesem Projekt verwendeten Implementierung prinzipiell gilt.

Implementierungstechnisch bestehen verschiedene Räume je nach Anwendungsfall, in denen Objekte durch Relativierung, Längenumrechnung und Transformation dargestellt werden.

\begin{enumerate}
\item Worldspace $= \mathcal{S}^3$ absolute Positionen von Objekten
\item Cameraspace $= \mathcal{F}^3$, Ursprung $O$ ist die Position der Kamera zum Rendern einer Szene, Objekte werden zur Kamera relativiert.
\item Viewspace $= \mathcal{F}^3$, Verzerrung durch die Kameralinse, um einen Blickwinkel auf einen Bildschirm anzupassen.
\item Objektspace,$= \mathcal{F}^3$, Ursprung ist der vom Modell definierten Mittelpunkt eines Objektes (Massenmittelpunkt), zur Verarbeitung von physikalischen Objektinteraktionen wird ein Objekt zu einem anderen Objekt relativiert.
\end{enumerate}

Auf diese Weise gelingt es selbst extreme absolute Entfernungen und Geschwindigkeiten im relativen akkurat zu behandeln.

\subsection{Entitäten}
\label{sec:entity}
Entitäten sind der atomare Inhalt der Simulation. Es handelt sich dabei um eine Abstraktion für \glqq Etwas\grqq  in der Simulation. In der Realität suchen Physiker immer weiter \glqq Das [sie] erkenne[n], was die Welt im Innersten zusammenhält, ..\grqq [- J.W.v. Goethe, Faust, \glqq Nacht\grqq, Vers. 382]. Was in der Realität eine offene Frage ist, obwohl seit den Zeiten von Goethe in der Quantenphysik durchaus Fortschritte gemacht worden sind, muss für die Simulation jedoch eindeutig beantwortet werden, um eine quantifizierbare Menge an Instanzen eines Konzepts zu haben, die dann Simuliert werden können.\\
Quantenmechanische Prozesse sind, um die Illusion einer realen Welt aus der Perspektive eines Menschen herzustellen, um einiges zu rechenaufwändig für unsere Anforderungen. Man setzt die Abstraktion der atomaren Simulationseinheit also höher an, was praktisch zu vielen Spezialisierungen führt. Die gemeinsame Basis dieser Spezialisierungen wird hier \textit{Entity}( dt. Entität ) genannt.\\

Verschiedenartige Beispiele:
\begin{enumerate}
\item klassisch: 3D-Objekt mit einer Position, Rotation und Form, welches sich in Abhängigkeit der Zeit mit einer Geschwindigkeit bewegt.\\
\item formlose: Entität kann z.B.~ein Geräusch sein, welches in der Simulation durch seinen Quellort abstrahiert ist, sich ebenfalls bewegen kann, aber keine Rotation besitzen muss, oder zugehörig zu einer anderen Entität ist.
\item unphysikalisch: Logische Entitäten, die beispielsweise die Anwesenheit eines Objektes in einem bestimmten Raumabschnitt prüft und einer Subroutine das Ergebnis übermittelt.
\end{enumerate}
Die Abstraktion der Entität dient demnach als Schnittstelle für die Simulation, um Zeit (während eines Ticks) auf atomare Bestandteile der Simulation anzuwenden, und als Implementierungsplattform für die verschiedenen Anwendungsanforderungen der Simulation, bzw.~dem Verhalten des Simulationsinhalts.

\subsection{Bounding-Volume}
Ein Bounding-Volume zu einem Objekt $o$ ist eine kompakte Menge $B_o \supseteq K_{o}$. $B_o$ kann als Hitbox fungieren.\\
Eine Bounding-Box ist ein spezielles Bounding-Volume in Form eines Quaders.\\
Eine in diesem Projekt extensiv verwendete, tiefere Spezialform der Bounding-Box ist die Axis-Alligned-Bounding-Box (AABB). Alle Kanten dieser Bounding-Box sind achsenparallel zu den Koordinatenachsen $\{(1,0,0), (0,1,0), (0,0,1)\}$ des 3D-Koordinatensystems.\\
Hier relevante Eigenschaften dieser sind:
\begin{itemize}
\item kleine Datenrepresentation:
		$$AABB_o \in \mathcal{S}^{3^2}; AABB_o = (BB_{min}, BB_{max}) = ((x_{min}, y_{min}, z_{min}), (x_{max}, y_{max}, z_{max}))$$.
		 In ihnen werden Minimal- und Maximalpositionen der AABB festgehalten.
		 Diese Positionen werden im Worldspace $\mathcal{S}^{3^2}$ angegeben, da AABBs hier die primäre vereinfachende Abstraktion sein sollen, die in der Simulation für Objekte verwendet wird. Die Angabe im Worldspace macht AABBs für Berechnungen im absoluten, zum Beispiel räumliche Partitionierung für Teile-und-Herrsche Algorithmen, direkt zugänglich.
		 Die theoretische, kompakte Punktemenge 
		 \begin{align*}
		 \mathcal{AABB}: \mathcal{S}^{3^2} \mapsto \mathcal{F}^3;
		 \mathcal{AABB} ((x_{min}, y_{min}, z_{min}), (x_{max}, y_{max}, z_{max})) = \\
		 \{meter((x_{min} + x * (x_{max} - x_{min}), y_{min} + y * (y_{max} - y_{min}),\\
		  z_{min} + z * (z_{max} - z_{min}))| x, y, z \in [0,1] \} 
		 \end{align*}
		 ist dann ein Bounding-Volume $\mathcal{AABB}(AABB_o) \supseteq K_o$
	\item Ermittlung einer minimalen AABB für ein Objekt durch Suche der Minima und Maxima der Ausdehnung eines Objektes in jeder Koordinatenachse:
	 $x_{min} = x : (x, y, z) \in V_o , \forall (x', y', z') \in V_o: x \leq x'; y \& z, min \& max $ analog.
	 Die Findung dieser Werte ist in $ \mathcal{O}(|V_o|) $.
		Bei rotierenden Objekten ist auch eine minimale AABB zu diesem Objekt einer Bewegung ausgesetzt, standardmäßig durch Neuermittlung der AABB($\mathcal{O}(n)$ pro Tick). Optimierungen für verschiedene Arten von Bewegung möglich (Positionsänderung, Skalierung, etc.), aber manchmal schwierig bis unmöglich (z.B. bei Rotation).\\
		Aus diesem Grund macht es auch Sinn, nicht-minimale AABBs zu wählen.
	\item Schnelle Kollisionsüberprüfung zwischen AABBs durch Vergleiche der Extrema $\mathcal{O}(1)$
\end{itemize}

AABBs, bzw. Bounding-Volumes generell, werden nicht ausschließlich für statische Objekte $K_{o,t}$ zu einem Zeitpunkt erstellt und verwendet. Es macht in bestimmten Kontexten zum Beispiel auch Sinn das komplette durchlaufene Volumen eines Objektes $\{K_{o,t'} | t' \in \Upsilon_{\delta_i}\}$ während eines Ticks durch ein Bounding-Volume zu abstrahieren.

Im Falle des Spiels Minecraft werden AABBs als finale Hitboxen verwendet (vgl. Abbildungen \ref{fig:mwhitbox}, \ref{fig:mphitbox}), welche jedoch scheinbar dem Kriterium $K_o \subseteq B_o$ zuwiderlaufen. Es muss an dieser Stelle zwischen der mathematischen Korrektheit einer Bounding-Box gegenüber einem gegebenen physikalischen Modell und der Designentscheidung gemacht werden, dass das sichtbare Modell nicht oder nur marginal die Grundlage des physikalischen Modells ist. In Minecraft ist die AABB die finale Hitbox  $H_{o, [min]} = K_o$ und definiert dadurch selbst das Modell $K_o$. Das Bounding-Box-Kriterium ist damit theoretisch erfüllt. Die Designentscheidung selbst soll an dieser Stelle nicht eingeschätzt werden.


\subsection{Interaktion}
Interaktionen werden zwischen Entitäten durchgeführt. Der Auslöser für eine Interaktion ist dabei an bestimmte Bedingungen geknüpft. Die wiederholte Auflösung der Bedingungen, sowie die Berechnung der konkreten Interaktion stellt einen wesentlichen Bestandteil der Aufgabe der Simulation dar.\\
Ein klassisches Beispiel einer Interaktion ist die physikalische Kollision zweier Objekte. Dabei muss u.U.~ der Hergang der Kollision, so wie eine passende Reaktion der Kollisionspartner ermittelt werden.\\
Als initiale Bedingung für das Auftreten einer Interaktion werden in dieser Simulation AABBs über dem Einflussgebiet einer Entität kollidiert, um entsprechend Interaktionspaare zu finden.

\subsection{Aktionen}
Aktionen werden von Entitäten durchgeführt und stellen eine unvorhersehbare Änderung einer Entität dar. Aktionen sind pro Entitätstyp definiert. Der Auslöser für eine Aktion ist dabei außerhalb der Definition der Entität zu suchen. Derzeit passt nur der menschliche Bediener auf diese Beschreibung, allerdings sind theoretisch auch andere Einflüsse, zum Beispiel durch eine Form von künstlicher Intelligenz denkbar, die nicht direkt als Teil der Entität modelliert wurde.


\subsection{Projektanteile/Neuleistungen}
Die im Abschnitt \ref{sec:terminology} behandelten Konzepte sind im Status-Quo zu Beginn des Projektes umgesetzt. Ihre Behandlung in diesem Bericht dient im Folgenden als Kontext.





