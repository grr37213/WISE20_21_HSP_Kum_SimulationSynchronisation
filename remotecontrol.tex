\section{Umsetzung der Fernsteuerung}

Die Fernsteuerung soll ausschließlich die Steuerung einer Spielfigur auf einem Remote ermöglichen. Ausgeschlossen sind dadurch die grafische Ausgabe an der Steuerungsapplikation und generell eine Rücksendung von Information vom Remote zur Steuerung.\\

\subsection{Formatierung von Benutzereingaben}
Zum Zweck der Fernsteuerung werden Benutzereingaben in ein übertragbares Format gebracht.\\
Es wird sich nach Abschnitt \ref{sec:transmission_formats} hier auf Grund der Echtzeitrelevanz auf die Übertragung von Statusinformation via UDP geeinigt.\\
Es wird zunächst in Betracht gezogen direkt Eingabeereignisse zu versenden. Eingabeereignisse sind
kleine Einheiten von Information, welche den Zustand oder die Änderung von Eingabegeräten angeben. Eingabeereignisse werden direkt von Bibliotheken ausgegeben, welche entsprechend Eingabegeräte dem Programm auf diese Weise zugänglich machen. Hier wird zu diesem Zweck SFML\cite{sfml} verwendet.\\
Allerdings entstehen hier einige Probleme:
\begin{itemize}
\item Eingabeereignisse bilden teilweise Stati und teilweise Änderungsinformation ab. Beispielsweise führt die Zustandsänderung einer Taste zu einem konsumierbaren Ereignis, was eine Änderungsinformation darstellt, welches selbst allerdings einen Status der Taste beinhält. Als ein anderes Beispiel existieren Ereignisse, welche die Differenz des Stands des Mausrades angeben, da dort keine absolute Skala verwendet wird. Die Probleme, welche aus diesen unterschiedlichen Formaten oder stattdessen der Verwendung von TCP entstehen wurden bereits in Abschnitt \ref{sec:transmission_formats} genauer beschrieben. Die Verwendung von rohen Eingabeereignissen erforderte also eine Umwandlung zu Statusinformation.
\item Eingabeereignisse sind teilweise clientseitig kontextbezogen. Vor allem die Verwendung von Ereignissen, welche von der Maus erzeugt werden, müssen oft unter Einbezug der Mausposition realisiert werden, welche relativ zur verwendeten Auflösung sowohl des verwendeten Monitors, als auch der Größe des Applikationsfensters abhängen. Beides sind Größen, auf welche die Applikation keinen authoritären Eingiff hat oder haben soll. Dazu kommen ebenfalls Probleme mit der Umsetzung verschiedener Maus-Modi, bei denen die Maus wahlweise als Zeigegerät oder als analoge Eingabemethode verwendet werden soll, was durch lokale Uminterpretierung der Mausereignisse realisiert wird.
\item Viele Eingabeereignisse stellen keine Information dar, welche den Remote, bzw. den Server tangieren. Größenveränderung des Client-Fensters, die Verwendung einer unbenutzten Taste, diese Dinge können als Information-Leakage interpretiert werden. Effektiv ist die Applikation unter diesen Umständen ein Keylogger.
\item Eingabeereignisse werden vielfach erzeugt und ändern oft nur marginal den Stand der Dinge. Es wird als verschwenderisch Angesehen eine Vielzahl gleichartiger Ereignisse unter Belastung der Verfügbaren Bandbreite zu versenden, wenn die meisten enthaltenen Ereignisse nur zu einer Überschreibung am Remote führen und effektiv keinen Effekt haben.
\item Um die verfügbare Bandbreite besser auszunutzen müssten Ereignisse gesammelt und in Gruppen versandt werden. Dadurch geht die Information über zeitliche Abstände zwischen Ereignissen weitestgehend verloren. Das Hinzufügen von Zeitstempeln zu Ereignissen wird als unnötiger Overhead sowohl für die Applikationen als auch die Übertragung angesehen.
\end{itemize}

Eingabeereignisse werden daher clientseitig in einen Kontrollstatus konsolidiert, eine Menge an Statusinformation, welche als Eingabeformat für die Remote-Simulation gilt, und zyklisch versendet werden kann. Die Größe dieser Menge ist vergleichsweise klein (derzeitig 64 Bytes).\\
Der Status umfasst ebenfalls nur Stati, welche für die Simulation von Interesse sind und bildet so die Bedienerintention gegenüber der Simulation, bzw. seiner Spielerfigur ab.\\
Einige Eingaben die ein Bediener tätigen kann, beziehen sich nicht auf Simulationsinhalte per se, sonders beispielsweise auf die Anpassung der Perspektive (erste und dritte Person) oder der Anzeige bestimmter Metriken der Simulation, z.B.~ Frames per Second, etc.. Diese können unabhängig vom Kontrollstatus clientseitig durch die Ereignisverarbeitung direkt verarbeitet werden.\\

Bei der Fernsteuerung ist zunächst die Rücksendung von Information vom Remote nicht erwünscht. Durch den Kontrollstatus als konsolidiertes Eingabeformat kann die unidirektionalität der Eingabe leichter versichert werden. Die Eingabe kann auf diese Weise zunächst auch erst einmal lokal getestet und verwendet werden, bevor die tatsächliche Fernsteuerung über Netzwerk umgesetzt wird.
Die Umsetzung einiger Features als unidirektionaler Status gestaltet sich jedoch nicht als trivial:
Zunächst waren zuvor einige Features ohne diese Anforderungen realisiert und erforderten einen Umbau, um einen unidirektionalen Status interpretieren zu können.
Außerdem sind einige Features strukturell unter den Anforderungen schwierig umzusetzen, bzw.~leiden sogar unter den Anforderungen:
\begin{itemize}
\item Die Betätigung eines Auslösers, z.B.~zum Schießen in der Shooter-Applikation ist starken Echtzeitanforderungen unterlegen. Bei längeren Latenzzeiten kann z.B. das klicken der Maus mehrfach geschehen. Wie oft und wie lange wurde der Schalter nun betätigt? Bei den möglicherweise anliegenden Latenzen ist eine dynamische Abtastrate durch einen nicht-dynamischen Status nicht abbildbar. Weiter ist die Abtastrate des Eingabegeräts ebenfalls beschränkt, wir schätzen diese allerdings als so hoch ein, dass dies irrelevant für unsere Überlegungen ist. An dieser Stelle sind unter den gegebenen Umständen die Fähigkeiten der Applikation beschränkt.
\item Die Selektion aus einer Menge von selektierbaren Dingen.\\
Ist die Menge der selektierbare Dinge bekannt und statisch, kann die Selektion durch einen einfachen Integer-Status  in einem Restraum dargestellt werden. Ist die Menge auszuwählender Elemente jedoch dynamisch, wird ohne die Möglichkeit der Rückkommunikation dieser Anzahl die Umsetzung dieses Features weitaus schwieriger, bzw. Anforderungen an den Selektionsmechanismus müssen grundlegend überdacht werden.
\end{itemize}

Einige Konzepte lassen sich jedoch vergleichbar leicht umsetzen.
\begin{itemize}
\item Absolute Position der Benutzerfigur
\item Absolute Ausrichtung der Benutzerfigur
\item Bewegungsintention des Bedieners
\item etc.
\end{itemize}

\subsection{Umsetzung im Design}

\begin{figure}
\centering
\begin{subfigure}[b]{0.3\textwidth}
\centering
\resizebox{\linewidth}{!}{
\begin{tikzpicture}[framed, thick,scale=1, every node/.style={scale=1}]

\begin{class}{Simulation}{0,-2}
\end{class}
\begin{class}{User}{0,0}
\end{class}
\begin{class}{LocalUser}{0,2}
\inherit{User}
\end{class}

\unidirectionalAssociation{User}{1}{}{Simulation}{1}{}

\end{tikzpicture}
}
\caption{Lokaler Anwender}
\label{fig:local}

\end{subfigure}
\begin{subfigure}[b]{0.3\textwidth}
\centering
\resizebox{\linewidth}{!}{
\begin{tikzpicture}[framed, thick,scale=1, every node/.style={scale=1}]

\begin{class}{Simulation}{0,-2}
\end{class}
\begin{class}{User}{0,0}
\end{class}
\begin{class}{RemoteControledUser}{0,2}
\inherit{User}
\end{class}
\begin{class}{RemoteControlSender}{0,4}
\end{class}

\unidirectionalAssociation{User}{1}{}{Simulation}{1}{}
\unidirectionalAssociation{RemoteControlSender}{1}{}{RemoteControledUser}{1}{}

\end{tikzpicture}
}
\caption{Ferngesteuerter Anwender}
\label{fig:remotecontrol_indiv}
\end{subfigure}
\caption{UML-Diagramme zur Umsetzung verschiedener Anwender}
\label{fig:remotecontrol_design}
\end{figure}

Unter dem neugewonnenen Design können verschiedene Arten von Benutzern definiert werden. Es wird der zunächst der lokale Benutzer definiert um den alten Verwendungszweck abzubilden, und dann eine neue Art Benutzer hinzugefügt, welcher einen Benutzer über eine Netzwerkverbindung abstrahiert. Man vergleiche hierzu die Abbildungen \ref{fig:remotecontrol_design}.\\
Die Abstraktion eines Benutzers kann in beiden Fällen das Eingabeformat des Kontrollstatus als gemeinsame Schnittstelle bereitstellen. Der Kontrollstatus wird je nach Art des Benutzers entweder lokal durch Ereignisse, oder über eine Netzwerkverbindung aktualisiert. 
In der Simulation wird pro Benutzer egal welcher Art eine Benutzerentität angelegt.
Diese werden dem Benutzer zugewiesen und sind fähig dessen Kontrollstatus entsprechend zu interpretieren.
