Echtzeitsynchronisation über Netzwerkverbindungen ermöglicht die Versicherung der Gleichheit zweier oder mehr sich auf verschiedenen Rechenmaschinen befindlichen Stati zu bestimmten Zeiten. Auf Basis dieses Verhaltens können bestimmte Nutzen geschlagen werden. Durch die Kenntnis des anderen Status kann das Verhalten auf einer entfernten Maschine lokal simuliert und weiterverarbeitet werden.Durch die Existenz eines lokalen Status, der zwischen mehreren Maschinen synchronisiert wird, können mehrere Bediener gleichzeitig in einem hohen Abstraktionsgrad Einfluss auf den Status nehmen.\\
Anwendungen für dieses Konzept lassen sich in beispielsweise in Videospielen finden, wo ein Simulationsstatus lokal angezeigt werden muss, ein Spieler jedoch auch selbst Einfluss auf das Geschehen hat. Weitere Beispiele lassen sich in der Robotik finden. Unabhängig davon ob ein Auto, Weltraumrakete oder Industrieroboter gebaut wird, Echtzeitsynchronisation ermöglicht Kommunikation zwischen einzelnen Bauteilen oder zwischen Maschine und Bediener in einer Weise, sodass die kommunizierte Information in geforderten Zeiträumen auf den jeweiligen Kommunikationsenden verfügbar ist.\\
In diesem Projekt soll, auf der Codebasis eines vorhergehenden Projektes aufbauend, Echtzeitsynchronisation erreicht werden. Der Status-Quo des Basisprojektes wird im Verlauf dieses Berichtes kenntlich gemacht.\\
Bei dem Projekt handelt es sich um eine 3D-Simulation von rigiden Objekten, welche sich bewegen. Ein Bediener bewegt sich ebenfalls im selben Raum, kann bestimmte Aktionen ausführen und nimmt somit auf den Status der Simulation Einfluss. Jeder Bediener soll dabei auch entsprechend graphische Ausgaben in Echtzeit erhalten.