

\subsection{Kontext und Motivation}
Eine Echtzeitsynchronisation einer Simulation auf mehrere über Netzwerk verbundene Rechenmaschinen soll die Äquivalenz der Simulationszustände auf allen beteiligten Maschinen zum selben, oder annehmbar ähnlichen Zeitpunkt gewährleisten.\\
Das Ziel ist die Kollaborative Verwendung der Simulation. 
Dazu zählen sowohl die Anzeige von Simulationsinhalten, als auch
die benutzerdefinierte Veränderung dieser. 
Die Synchronisation der Simulationszustände beinhaltet demnach die rechtzeitige Synchronisation der Änderungen.

Nach dieser Definition oder nach dessen Teilen, finden wir Echtzeitsynchronisation in industriellen Anwendungen wieder. Zum Beispiel kann im Bereich der Robotik der rechtzeitige Informationsfluss zwischen Steuergeräten und Sensoren als Echtzeitsynchronisation angesehen werden.
%%TODO cite example
Das wohl ähnlichste Anwendungsbeispiel im Kontext dieser Arbeit lässt sich im Bereich der Videospiele finden, wo ein echtzeitsynchronisierter Simulationsstatus auf mehreren Maschinen lokal angezeigt werden muss, Bediener jedoch auch selbst Einfluss auf die Simulationsinhalte nehmen können.\\
In diesem Projekt wird auf einer bestehenden Codebasis einer Simulation, welche nicht unter Einbezug von Synchronisationsgedanken entwickelt wurde, eine solche umgesetzt. Dabei soll eine Echtzeitsynchronisation zwischen mehreren Rechenmaschinen über eine Internetverbindung mit 100\,KByte/s Bandbreite erreicht werden.
Das letztendliche Ziel ist die möglichst unmerkliche Synchronisierung zweier Simulationszustände und deren Anzeige, d.h. die Reduzierung von Seiteneffekten durch Latenz.\\
Im Laufe der Arbeit werden Kernaspekte zur Synchronisation, aber auch zum benötigten Softwaredesign ermittelt, um synchrone Applikationen umzusetzen. Durch den Status-Quo der übernommenen Codebasis werden durch die Überarbeitung dort gemachte Fehler in diesen Aspekten erkenntlich.

\subsection{Beschreibung der konkreten Simulationsanwendung}
Die verwendete Simulationsanwendung ist ein rudimentäres 3D-Shooter-Videospiel. Ein Spieler kann dabei in einem Terrain auf Gegnerfiguren schießen, diese beschädigen und wird dabei durch Punkte entlohnt. Diese einfache Anwendung hält bereits alle Kernaspekte einer Simulation im Kontext dieser Arbeit inne:
\begin{itemize}
\item Die Simulation ist gezeitet und läuft in einer bestimmten Rate zur Realzeit. Diese Rate soll meist $1$ sein, bei der programmierte physikalische Vorgänge dieselbe Geschwindigkeit wie in der Realität annehmen. Die Simulation besitzt dadurch ihre eigene Simulationszeitbasis.
\item Verschiedene Arten von zu simulierenden Entitäten in einem Simulierten Raum (Gegner, Spielerfiguren, Projektile, Geräusche) mit unterschiedlichem Verhalten \/unterschiedlicher Physik.
\item Interaktionen zwischen Entitäten (Kollisionen, Anti-Clipping)
\item Einflüsse durch Bediener/Spieler (Bewegung der Spielerfigur, Erzeugung von Projektilen)
\item Grafische Ausgabe in Echtzeit zu einer dreidimensionalen Perspektive
\end{itemize}

Die Simulation von Simulationsinhalten erfolgt in Schritten, in denen ein Inhalt von einem bestehenden Zustand auf einen Zustand zu fortgeschrittener Simulationszeit verändert wird. Ein solcher Zeitschritt wird als Tick bezeichnet (siehe Appendix \ref{sec:tick} oder vgl.~Quelle \cite{tick}).\\
Die Anzeige der Simulation erfolgt durch Extrapolation des letzten bekannten Zustands der Simulationsinhalte. Die Anzeige führt dabei keine Änderungen auf den Simulationsinhalten durch. Wir bezeichnen den Prozess sowie sein Ergebnis als Frame (vgl. Appendix \ref{sec:tick}).
Die Framerate kann ungleich der Tickrate sein, um unabhängig von den Fähigkeiten des Anzeigesystems eine flüssiges Bild zu generieren. Da in einem Frame keine Simulationsinhalte geändert werden, sind die Simulationinhalte komplett unabhängig zur Anzeige.\\

Die beschriebene Anwendung funktioniert zu Beginn des Projektes lokal mit einem Benutzer. Es ist Ziel dieser Arbeit möglichst volle Funktionalität auf mehrere Benutzer, welche mit einer Netzwerkverbindung verbunden sind zu erweitern und dabei möglichst wenige negative Seiteneffekte zu erzeugen.

\subsection{Hergang}
Der Entwicklungsprozess in diesem Projekt umfasst das Erreichen bestimmter Meilensteine, die konstruktiv auf das Endziel hinführen sollen. Dieser Vorgang wird gewählt, um die Gesamtkomplexität des Endziels zu mitigieren. Durch die Abhängigkeit, die das Projekt vom Status quo des Vorprojektes hat, kann so dieses sukzessive an die hier gestellten Anforderungen angepasst werden.\\
Der Hergang lautet wie folgt:
\begin{enumerate}
\item Aufarbeitung des Status Quo\\
Umsetzung eines Designs im bestehenden Basisprojekt, das mit weiteren in diesem Projekt zu entwickelnden Features kompatibel ist.
\item Implementierung einer Fernbedienung\\
Fernbedienung als erstes netzwerkabhängiges Feature mit Einwegkommunikation, die keine Synchronisation benötigt.
\item Erneute Aufarbeitung
\item Implementierung der Echtzeitsynchronisation zwischen Simulationen\\
Auf zwei über Netzwerk verbundenen Maschinen wird jeweils eine Simulationsinstanz ausgeführt, deren Inhalte Synchronisiert werden.
\end{enumerate}

