Der Entwicklungsprozess in diesem Projekt umfasst das Erreichen bestimmter Meilensteine, die konstruktiv auf das Endziel hinführen sollen. Dieser Vorgang wird gewählt, um die Gesamtkomplexität des Endziels zu mitigieren. Durch die Abhängigkeit, die das Projekt vom Status quo des Vorprojektes hat, kann so dieses sukzessive an die hier gestellten Anforderungen angepasst werden.\\
Der Hergang lautet wie folgt:
\begin{enumerate}
\item Aufarbeitung des Status Quo\\
Umsetzung eines Designs im bestehenden Basisprojekt, das mit weiteren in diesem Projekt zu entwickelnden Features kompatibel ist.
\item Implementierung einer Fernbedienung\\
Fernbedienung als erstes netzwerkabhängiges Feature mit Einwegkommunikation, die keine Synchronisation benötigt.
\item Erneute Aufarbeitung
\item Implementierung der Echtzeitsynchronisation zwischen Simulationen\\
Auf zwei über Netzwerk verbundenen Maschinen wird jeweils eine Simulationsinstanz ausgeführt, deren Inhalte Synchronisiert werden.
\end{enumerate}

