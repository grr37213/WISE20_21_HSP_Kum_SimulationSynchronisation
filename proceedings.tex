\label{sec:proceedings}

Der Entwicklungsprozess in diesem Projekt umfasst das Erreichen bestimmter Meilensteine, die konstruktiv auf das Endziel hinführen. Dieser Vorgang wird gewählt, um die Gesamtkomplexität des Endziels zu abzuschwächen und Bearbeitungszeiten einzuschätzen. Durch die Abhängigkeit, die das Projekt vom Status quo des Vorprojektes hat, kann so dieses sukzessive an die hier gestellten Anforderungen angepasst werden.\\
Der Hergang lautet wie folgt:
\begin{enumerate}
\item Implementierung einer Fernbedienung\\
Eine Fernbedienung als erstes netzwerkabhängiges Feature mit Einwegkommunikation zur Realisierung von Steuerungsmechanismen, die keine Synchronisation des Simulationsinhalts erfordert.
\item Implementierung der Echtzeitsynchronisation zwischen Simulationen\\
Auf zwei über Netzwerk verbundenen Maschinen wird jeweils eine Simulationsinstanz ausgeführt, deren Inhalte aktiv synchronisiert werden.
\item Aufarbeitung der im Projekt enthaltenen Features, um eine verwendbare Applikation umzusetzen: Ein rudimentärer Videospiel-Shooter, bei dem nun mehrere menschliche Teilnehmer möglich sind.
\item Weitere Anpassungen, welche sich zum Vorteil der Applikation unter den neuen Umständen auswirken, um Seiteneffekte der Netzwerksynchronisation zu verringern.
\end{enumerate}

