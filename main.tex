\documentclass[11pt,twoside,a4paper]{article}

\usepackage{a4wide,amsmath,amssymb}

% Mann will direkt Umlaute eingeben können statt \"a, \"o, \"u usw.
% Entweder:
\usepackage[utf8]{inputenc}
% oder:
%\usepackage{umlaut}
\usepackage[german]{babel}

\usepackage[style=numeric]{biblatex}
\addbibresource{grr.bib}

\usepackage{textcomp}
\usepackage{graphicx}
\usepackage{subcaption}

\usepackage{hyperref}

\usepackage{tikz}


% Trennvorschl"age (in {} einfuegen, wenn nicht automatisch getrennt wird:
% z.B. Authen-ti-ka-tions-sys-tem)
%\hyphenation{}

%\hyphenation{min-des-tens}
%\hyphenation{Kol-li-sions-er-ken-nung}


%-------------------------- Formatsachen --------------------------%

% Bild-, Tabellenunterschriften veraendern:
% Nummer fett, kleinerer Text fuer Bildunterschrift
%\usepackage[bf,small]{caption}


%\usepackage{mathpazo}  % -- Palatino als Zeichensatz -- einfach diese
					   % Zeile auskommentieren, falls nicht installiert
%\usepackage{mathptmx}  % -- Times als Zeichensatz

% Zum Unterscheiden von Entwurfs- und endgueltiger Fassung
%\usepackage{draftcopy}
%\draftcopySetGrey{0.90}   %   90% = sehr helles Grau
%\draftcopyName{ENTWURF}{155}   % statt ``DRAFT''
%\draftcopySetScale{1}

%--------------- Zeilen- und Absatzabstaende ----------------------%
%\setlength{\parindent}{0em}
%\setlength{\parskip}{\medskipamount}    % Abstand zwischen Abs"atzen


\begin{document}

\title{HSP-Projektarbeit im Master Informatik \\
\small Echtzeitsynchronisation von Simulationen mit Fokus auf die Verwendung in Videospielen}
\author{Robert Graf, Lukas Hermann\\
%  (\texttt{fridolw@in.tum.de})\\[5mm]
%  Seminar "`Internetrouting"' , \\
  Ostbayerische Technische Hochschule Regensburg\\
  \\
  Projektbetreuung: Prof. Dr. Markus Kucera
}
  
\date{WS\, 2019/2020 (Version vom \today)}

\maketitle

\newpage
\tableofcontents


\abstract{Das Projekt umfasst die Erstellung einer 3D Simulation mit dem Fokus auf die Umsetzung von Echtzeitsynchronisierung besagter Simulation. In diesem Bericht werden sowohl behandelte Themen als auch vollzogene T"atigkeiten und H"urden in der Bearbeitung dargestellt.}


\section{Einleitung}
Eine Echtzeitsynchronisation einer Simulation auf mehrere über Netzwerk verbundene Rechenmaschinen soll die Äquivalenz der Simulationszustände auf allen beteiligten Maschinen zum selben Zeitpunkt gewährleisten.\\
Die Synchronität ermöglicht die Verwendung der Simulation auf mehreren Netzwerkknoten, übermittelt die enthaltene Information in der Simulation und ermöglicht Einwirkungen auf deren Inhalt in Kollaboration mit mehreren Benutzern.
Aspekte der Echtzeit sind dahingehend ausschlaggebend, dass unabhängig davon ob ein
herkömmlich simulierter Simulationsinhalt durch simulierte Interaktion oder ein Benutzer der Verursacher von Simulationsvorgängen ist, trotzdem synchron auf allen beteiligten Maschinen abläuft.\\

Eine Form der Echtzeitsynchronisation wird im Bereich der Robotik eingesetzt, um eine Echtzeitanforderung zu realisieren. Echtzeitsynchronisation behandelt dabei meist den rechtzeitigen Informationsfluss zwischen Steuergeräten und Sensoren, sodass diese ihre Aufgabe auf Basis wahrer und aktueller Informationen optimal durchführen können.
%%TODO cite example
Das wohl ähnlichste Anwendungsbeispiel im Kontext dieser Arbeit lässt sich in Videospielen finden, wo ein echtzeitsynchronisierter Simulationsstatus lokal angezeigt werden muss, ein Spieler jedoch auch selbst Einfluss auf das Geschehen hat.\\
In diesem Projekt wird auf einer bestehenden Codebasis einer Simulation, welche nicht unter Einbezug von Synchronisationsgedanken entwickelt wurde, eine solche umgesetzt. Dabei soll eine Echtzeitsynchronisation erreicht werden.
Das letztendliche Ziel ist die möglichst unmerkliche Synchronisierung zweier Simulationszustände und deren Anzeige.\\
Im Laufe der Arbeit werden Kernaspekte zur Synchronisation, aber auch zum benötigten Softwaredesign ermittelt, um synchrone Applikationen umzusetzen. Durch den Status-Quo der übernommenen Codebasis werden durch die Überarbeitung dessen gern gemachte Fehler erkenntlich.

\section{Kontext \& Terminologie}
Implementierungstechnisch schränkt eine Rechenmaschine die mathematischen Zahlenräume ein:
\begin{itemize}
\item Die Reelen Zahlen $\mathbb{R}$ beschränken sich auf Floating-Point-Datentypen, welche hier im weiteren mit $\mathcal{F} \subset \mathbb{R}$ bezeichnet werden
\item Integern $\mathbb{Z}$ sind maschinell in ihrer Darstellungsgröße beschränkt. Diese beschränkte Menge an Integern wird $\mathcal{I} \subset \mathbb{Z}$ genannt
\end{itemize}


\subsection{Zeit}
\label{sec:time}
\def\finite#1{\ooalign{\hfil$\mapstochar\mkern 3mu\mapstochar\mkern 5mu$\hfil\cr$#1$}}

Die Simulation läuft gezeitet ab. Es existieren dabei zwei relevante zeitliche Sequenzen:
\begin{enumerate}
\item Die durch maschinelle Abtastung diskrete Realzeit der echten Welt\\
	\begin{itemize}
	\item $T_r:=\langle t_{epoch}, ... , t_{max}\rangle ; t_{epoch}, t_{max}$ als minimal, bzw.~maximal darstellbare Zeit
	\item die in einer maschinellen Genauigkeit in Mikrosekunden $$\epsilon_t:=10^{-6}s ; \forall c \in \mathbb{Z}: T_r(c) + \epsilon_t = T_r(c+1)$$ gegeben ist
	\item und immer monoton wächst $\forall c \in \mathbb{Z}: T_r(c) < T_r(c+1)$. 
	\end{itemize}

\item Die Simulationszeit $T_s:=\langle t_{start}, ... , t_{end}\rangle; t_{start}, t_{end}$ als Start- und Endzeitpunkt der Simulation.
	\begin{itemize}
	\item Zwischen den beiden Zeitbasen besteht eine totale, nicht-injektive, surjektive Abbildung $\mathcal{T}:T_r \twoheadrightarrow T_s$
	\item Die Simulationszeit ist dadurch relativ zur Realzeit definiert $T_s:=\langle\mathcal{T}\rangle$
	\item Um die Kontinuität der Zeit herzustellen wird weiter eine Zeittate $r_t:T_r\mapsto\mathcal{F}$ definiert, welche das relative verstreichen der Zeit in der Simulation steuert. 
	$$\forall t_{r0},  t_{r1} \in T_r ; t_{diff}=t_{r1}-t_{r0} :\mathcal{T}(t_{r1}) = \mathcal{T}(t_{r0}) + t_{diff}*r_t(t_{r1})$$ unter der Vorraussetzung, dass die Rate während aller Zeiten zwischen $t_{r0}$ und $t_{r1}$ gleich bleibt $\forall t \in [ t_{r0},t_{r1}]: r_t( t_{r0}) = r_t(t)$. Wird die Eigenschaft des Zeitflusses in der festgelegten Rate verletzt, werden die aktuellen Echtzeitanforderungen verletzt. \\
	Soll die Rate also geändert werden muss dies zu festgelegten Umschaltpunkten geschehen, welche die Simulationszeit in Zeitbereiche trennen, zwischen denen keine Berechnungsvorgänge zuverlässig durchführbar sind.\\
Beispiele für Raten sind 
\begin{itemize}
\item $r_t(x) = 1 \Leftrightarrow$ Simulation synchron zur Echtzeit
\item $r_t(x) = 0 \Leftrightarrow$ Simulation ist pausiert/läuft nicht
\end{itemize}
Technisch wird für große $|r_t|$ die Simulation schwierig, da viele Vorgänge schnell simuliert werden müssten. Diese werden daher vermieden.\\
Theoretisch kann die Rate auch negative Werte annehmen. Die Simulationszeit würde dann rückwärts laufen. Dieses Verhalten ist technisch durch die monoton steigende Realzeit nicht leicht in konsistenter Weise umzusetzen, da Realzeiteinflüsse durch Tasteneingaben existieren und soll daher hier ebenfalls vermieden werden.
	\end{itemize}
\end{enumerate}	


\subsection{Tick \& Frame}

Die Simulation behandelt das Verstreichen von Zeit in Zeitschritten, während dem der interne Zustand der Simulation, bzw.~der simulierten Objekte, zu einem zeitlich neuen Zustand aktualisiert wird.
Dieser Zeitschritt, bzw.~Verarbeitungsschritt, wird oft als Tick bezeichnet.\\

Es ist besonders anzumerken, das der Begriff des Ticks sich ausschließlich auf das Voranschreiten der Simulation bezieht und nicht dem Anzeigen einer Szene. Die Äquivalente Bezeichnung im Kontext der Grafik wird als Frame bezeichnet, in welchem eine Szene (ein Grafischer Zustand der Simulation) gerendert wird. Es besteht Verwechslungsgefahr. Von beiden Größen können Raten $r_{tick}, r_{frame}$ angegeben werden (üblicherweise in Tick/Frame pro Realzeitsekunde). Praktisch kann eine Grafikengine durch Inter- oder Extrapolation unterschiedliche Raten erreichen.\\
Wir definieren die Menge der Ticks $\delta:T_r^2; \delta:=\{\delta_1, \delta_2, ...\}$ anhand ihrer Start- und Endzeitpunkte in Echtzeit $\delta_i := (\delta_{i0}, \delta_{i1})$ und erweitert die zu einem Tick gehörenden Zeitpunkte als $\delta_{id}; d \in [0,1]$. Die Inklusivität/Exklusivität muss in bestimmten Berechnungskontexten manchmal angepasst werden um bei sukzessiven Ticks doppelte Behandlungen von Ereignissen zu vermeiden. Diese Einschätzung sei für jeden Kontext individuell zu vollziehen.\\
Es gilt außerdem die Kontinuität der Zeit auch bei Ticks $\delta_{j1} = \delta_{(j+1)0}$, d.h. ein Tick beginnt am Endzeitpunkt des Vorherigen.\\
Durch die Abbildung $\mathcal{T}$ erhält der Tick eine Entsprechung in Simulationszeit.\\
Ist im aktuellen Kontext nur ein Tick $\delta_i$ von belang wird auch die Terminologie $t_d =\mathcal{T}(\delta_{id})$, also $t_0$ für die Tickbeginn und $t_1$ für das Ende in Simulationszeit, verwendet.\\
Man kann weiter die Sequenz $\Upsilon_{\delta i} = \langle t_0, ...,  t_1\rangle$ als die zusammenhängende Partition der Simulationszeitsequenz $T_s$ denotieren, welche die geordneten Zeitpunkte eines Ticks in Simulationszeit enthält.\\
Die in Abschnitt~\ref{sec:time} beschriebenen möglichen Umschaltzeiten zur Änderung der Zeitflussrate in der Simulation werden auf die Grenzen von Ticks gelegt.\\
Die Größe der Zeitdifferenz $t_1 - t_0$ unterliegt meist Einschränkungen. Bestimmte Simulationsalgorithmen wie z.B. die Methode der kleinen Schritte erfordert für eine bestimmte Genauigkeit eine maximale Schrittgröße. Die verfügbare Rechenleistung hingegen beschränkt die Tickrate nach oben. Reicht die Berechnungszeit während eine Ticks nicht um den Status der Simulation von $t_0$ auf $t_1$ zu aktualisieren, läuft die Simulation langsamer als die reale Zeit. Die Echtzeitanforderung ist dann verletzt. Oft wird die Tickrate als Konstante festgelegt, in diesem Projekt ist jedoch nur eine Mindestrate festgelegt.


\subsection{Raum}
\label{sec:space}
Der geforderte 3D Raum kann durch 3-dimensionale Vektoren $\in \mathcal{F}^3$ in der Einheit Meter beschrieben werden.\\
Durch die Werteverteilung in $\mathcal{F}$ treten jedoch bei großen Räumen für Positionen mit großer Entfernung zum Ursprung $O$ Genauigkeitsdefizite auf, die zur Verletzung von Genauigkeitsanforderung führen können. Mögliche Floating-Point-Werte liegen dabei dichter beieinander, je näher am Ursprung \cite{floatdistribution}. Ein Beispiel für die Auswirkungen dieses Sachverhalts in Simulationen kann in der Quelle \cite{floatdistributionexample} betrachtet werden.\\
Physikalische Prozesse berechnet auf Basis von Positionen in $\mathcal{F}^3$ können daher inkonsistent in Abhängigkeit zum Ort im Raum sein.\\
Das Problem wird hier durch einen neuen Längendatentypen $\mathcal{S} : \mathcal{I} \times \mathcal{F}$ gelöst, welcher den Raum zunächst gleichmäßig durch $\mathcal{I}$ aufteilt und indiziert und $\mathcal{F}$ als Offset innerhalb seines Raumteils verwendet. Es wird daher eine Größe der initialen Aufteilung $size_{grid}$ definiert.\\
Die Umrechnung zu Metern ist dann: $$ meter: \mathcal{S} \mapsto \mathcal{F};  meter((i, f)) = i * size_{grid} + f * size_{grid}$$ 
Typischerweise gilt $f \in [0;1[$, um eine eindeutige Repräsentation für einen beschriebenen Punkt zu erhalten.

Diese Darstellung hat folgende weitere Vorteile
\begin{itemize}
\item Einfache Implementierung
\item Schnelle Indizierung der durch $\mathcal{I}$ indizierten Raumanteile für raumaufteilende Teile-und-Herrsche-Algorithmen
\end{itemize}

Absolute Positionen im Raum werden demnach mit Vektoren $s\in\mathcal{S}^3$ dargestellt. Für Berechnungen von Interaktionen zwischen Objekten werden Positionen zunächst relativiert, d.h. Positionen $p \in \mathcal{S}$ sollen zu $p_0$ relativ gesetzt werden, dann sind die relativen Positionen $p' = p - p_0$. Diese werden dann in in die für lokale Interaktionen sinnhafte  Einheit Meter $\mathcal{F}^3$ umgewandelt, um darauf Berechnungen durchzuführen. Man geht dabei davon aus, das relative Strecken zwischen Objekten kurz genug sind, sodass die Genauigkeitsänderung in $\mathcal{F}$ vernachlässigbar ist.\\
Effektiv ist dabei die Eigenschaft $\mathcal{F}\subset\mathcal{S}$ nicht gefordert, auch wenn sie in der in diesem Projekt verwendeten Implementierung prinzipiell gilt.

Implementierungstechnisch bestehen verschiedene Räume je nach Anwendungsfall, in denen Objekte durch Relativierung, Längenumrechnung und Transformation dargestellt werden.

\begin{enumerate}
\item Worldspace $= \mathcal{S}^3$ absolute Positionen von Objekten
\item Cameraspace $= \mathcal{F}^3$, Ursprung $O$ ist die Position der Kamera zum Rendern einer Szene, Objekte werden zur Kamera relativiert.
\item Viewspace $= \mathcal{F}^3$, Verzerrung durch die Kameralinse, um einen Blickwinkel auf einen Bildschirm anzupassen.
\item Objektspace,$= \mathcal{F}^3$, Ursprung ist der vom Modell definierten Mittelpunkt eines Objektes (Massenmittelpunkt), zur Verarbeitung von physikalischen Objektinteraktionen wird ein Objekt zu einem anderen Objekt relativiert.
\end{enumerate}

Auf diese Weise gelingt es selbst extreme absolute Entfernungen und Geschwindigkeiten im relativen akkurat zu behandeln.

\subsection{Objektform}
\label{object_form}
Die Form eines Objektes ist in einem hier rigiden, d.h. unveränderlichen Modell beschrieben, welche die Form relativ zum Ursprung $O$ ihres eigenen Objektraums angibt.
Für Modelle werden mathematisch oft als kompakte Punktemengen verwendet. Wir denotieren diese kompakten Modelle zugehörig zum Objekt $o$ als $ K_o \subseteq \mathcal{F}$.\\

Durch die kompakte Mengendarstellung führen Rechenoperationen mit Objekten auf maschinell relativ rechenaufwändige Mengenoperationen zurück. In der Computergraphik werden deshalb Objekte durch sogenannte Polygon-Meshes dargestellt. 
Für das Objekt $o$ ist eine Polygon-Mesh $M_o := (V_o, I_o); V \subseteq \mathcal{F}^3, I \subseteq [0, |V|-1]_\mathbb{N}^3 )$ beschreibt ein Polytop durch seine Eckpunkte $V_o$(Ecken, eng. \textit{vertices}), die durch 3-Gruppierungen ihrer Indices $I_o$ zu Dreiecksflächen verbunden sind.\\
Aus der Definition der Polygon-Mesh gehen implizit weitere Definitionen hervor:

\begin{enumerate}
\item Kanten (eng. \textit{edges}) $E_o = \{(v_a, v_b), (v_b, v_c),(v_c, v_a) | \{v_0, ...\} = V_o;(a, b, c) \in I\} $\\
, welche jeweils die Punkte $\mathcal{E}:E_o\mapsto\mathcal{F}^3; \mathcal{E}((v_a, v_b)) = \{v_a + (v_b-v_a)* k; k \in [0,1]\} $ im Raum einnehmen.
\item Flächen (eng. \textit{areas})$ A_o = \{(v_a, v_b, v_c) | \{v_0, ... \} = V(o, t); (a, b, c) \in I\} $,\\
welche jeweils die Punkte $\mathcal{A}:A_o\mapsto\mathcal{F}^3; \mathcal{A}((v_a, v_b, v_c)) = \{v_a + (v_b-v_a)* k + (v_c-v_a)*l; (k+l) \in [0,1]\} $ im Raum einnehmen.
\item Gesamtpunktemenge $G_o = V_o \cup (\bigcup_{e\in E_o} \mathcal{E}(e)) \cup (\bigcup_{a\in A_o} \mathcal{A}(a)) $
\item Praktisch immer gilt: $V_o \subset (\bigcup_{e\in E_o} \mathcal{E}(e)) \subset (\bigcup_{a\in A_o} \mathcal{A}(a)) = G_o \subset K_o$
\item Die Gesamtpunktemenge enthält zu allen Zeiten $t$ mindestens die Objekthülle $\mathcal{H}: \mathcal{F}^3 \mapsto \mathcal{F}^3, \forall o\in OBJ: \mathcal{H}(K_o) \subseteq G_o$ die den eingenommenen Raum des Objekts/Modells vom übrigen Raum durch Flächen abgrenzt.
\end{enumerate}

Vorteile \& Nachteile dieser Darstellung o.B.d.A:
\begin{itemize}
\item [+]Kürzere Iterationslängen im Vergleich zu kompakten Punktmengen: $V_o \ll G_o \ll K_o$
\item [+]Berechnungen durch relativ schnelle klassische Vektorarithmetik, Skalar- , Kreuzprodukte statt Mengenoperationen.
\item [-]Verlust der Information von Innen \& Außenseite am Hüllobjekt.
\item [-]Zum sinnvollen Einsatz von Polygon-Meshes ist die verwendbare Menge an Objekten auf Polytope beschränkt. Eine schwierig darstellbare Objektform sind beispielsweise Ellipsoide ($|V_o|$ geht dann gegen $|G_o|$).
\end{itemize}

Während semantisch von der Simulation die kompakte Repräsentation eines Objektes $K_o$ respektiert werden muss, rechnet diese tatsächlich mit gegebenem $G_o$ welches in den meisten Kontexten genügt.

\subsection{Objektplatzierung}
\label{sec:objects_sim}

Objekte $o\in OBJ$ müssen nun in der Simulation absolut im Worldspace $\mathcal{S}^3$ platziert werden, sollen sonst für Berechnungen aber in relativen Räumen $\mathcal{F}^3$ behandelt werden. Es werden dafür zunächst absolute Beschreibungskriterien für Objekte angelegt.
\begin{itemize}
\item Raumposition zu Beginn eines Ticks $pos : OBJ \times \delta \mapsto \mathcal{S}^3$
\item Ausrichtung zu Beginn eines Ticks $rot : OBJ \times \delta \mapsto \mathcal{F}^3$. Der Vektor $(x, y, z) \in\mathcal{F}^3$ wird für die Drehung um x (Radialmaß) für die Drehung um die x-Achse relativ zum Raum definiert, bzw. y und z analog. Andere etablierte Formate, wie Quaternionen, werden hier nicht verwendet.
\end{itemize}
Die Kontinuitätseigenschaft $\delta_{j1} = \delta_{(j+1)0}$ ermöglicht die Übernahme/Speicherung des Wertes zu Tickbeginn aus dem letzten Tick.

Objekte sind in der Simulation zusätzlich zeitlichen Änderungen unterlegen.
An dieser Stelle wird festgelegt: Während eines Ticks ändern sich diese konstanten zeitlichen Änderungsgrößen nicht und werden daher ebenfalls pro Tick definiert.\\
 Es wird sich hier auf
\begin{enumerate}
\item Geschwindigkeit $v: OBJ \times \delta \mapsto \mathcal{S}^3$  und
\item Winkelgeschwindigkeit $\omega : OBJ \times \delta \mapsto \mathcal{F}^3 $
\end{enumerate}
beschränkt.

Während eines Ticks $\delta_i$ kann demnach eine Transformationsmatrix $Q: OBJ \times \Upsilon_{\delta_i} \mapsto \mathcal{F}^{4\times 4}$ für alle für jedes Objekt zu jedem Zeitpunkt des Ticks angegeben werden, die die einem Objekt via Modell zugeordneten Punkte, seien dies $V_o, K_o$ usw., die relativ zum jeweiligen Objektspace angegeben sind nun in die entsprechende Position im Worldspace transformieren kann. Transformationsmatrizen sind typischerweise im Kontext der Computergrafik und Simulation von 3D-Räumen $4\times 4$ gewählt, um z.B. auch Translation zu realisieren \cite[ch. 4.4.1, p.76]{fourcrossfour}. Wir beschreiben dazu die 
Translationsmatrix, die aus einer Position, und die Rotationsmatrix, die aus einer Rotationsanweisung hervorgeht $Q_{trans}, Q_{rot}:\mathcal{F}^3 \mapsto \mathcal{F}^{4\times 4}$. Beide dieser Funktionen, insbesondere $Q_{trans}$ sind für $\mathcal{F}^3$ anstatt für $\mathcal{S}^3$ definiert, denn es wird erwartet, dass für die Berechnung zu einem relativ nahen Punkt $P \in\mathcal{S}^3$ relativiert und daraufhin in $\mathcal{F}^3$ durch die Funktion $meter$ umgewandelt wird.\\
Sei $o \in OBJ; pos = pos(o, \delta_i); v = v(o, \delta_i); rot = rot(o, \delta_i); \omega = \omega(o, \delta_i)$, dann gilt
$Q(o, t) = Q_{trans}(meter(pos + v * (t-t_0))) * Q_{rot}(rot + \omega * (t - t_0))$.\\
Bei der Relativierung von Objekten ,welche hier als $o_1 - o_0$ denotiert wird, werden die Beschreibungskriterien beider Objekte mit denen eines Objektes $o_0$ jeweils subtrahiert. Die relative Transformationsmatrix 
\begin{align}
R: OBJ^2 \times \Upsilon_{\delta_i} \mapsto \mathcal{F}^{4\times 4}; R(o_1, o_0, t) = Q(o_1 - o_0, t) = \\
Q_{trans}( meter(  pos(o_1, t)-pos(o_0, t) ) + meter(v(o_1, t)-v(o_0, t)) * (t-t_0) ) \\
* Q_{rot}((rot(o_1, t)-rot(o_0, t)) + (\omega(o_1, t)-\omega(o_0, t)) * (t-t_0))= I_4
\end{align}
. Ein Vorteil besteht dabei, dass bei Relativierung eines Objektes zu sich selbst dabei die Identitätsmatrix $I_4 = R(o_0, o_0, t) \forall t$ entsteht. Auf entsprechenden Modellpunkten von $o_0$ muss dann keine Operation ausgeführt werden, da $\forall p\in \mathbb{F}^3: p*I_4=p$. Dadurch kann bei der Berechnung einer Interaktion, bei der u.U.~ viele Modellpunkte Transformiert werden müssen, die Hälfte der Transformationen gespart werden.\\
Zur Vereinfachung wird für alle Punktmengen $X_o$ zu einem Objekt $o$ die Notation $X_{o, t} = X_o * Q(o, t)$ und für relative Transformationen $X_{o_1, t, o_0} = X_{o_1} * R(o_1, o_0, t)$ verwendet, sodass z.B. die Transformierte Menge der Eckpunkte von $o$ zu Tickzeitpunkt $t$ als $V_{o,t}$ beschrieben werden kann.

\subsection{Projektleistung}
Im Rahmen des Projektes wurden Schnittstellen für~\ref{sec:objects_rep}, angelegt um entsprechenden Zugriff auf die Information von Ecken, Kanten und Flächen während Interaktionsroutinen zu erhalten. Außerdem wurden physikalische Entitäten im Code Repräsentiert, die die für die paarweise Kollisionsberechnung benötigten Attribute für Rotation und Rotationsgeschwindigkeit als Erweiterung zur herkömmlichen Entität beinhalten, die diese nicht besitzt. Alle anderen Teile des bisher behandelten Kontext von Zeit, Raum Objektrepräsentation und Objektbewegung enthalten im Rahmens dieses Projektes maximal marginale Neuleistung.




\subsection{Hitbox}
\label{sec:hitbox}
Hitbox $H_o \approx K_o$ ist eine Approximationen für ein Modell in einer Simulation. Sie abstrahieren das konkrete Modell dabei für einige oder gar alle physikalischen Berechnungskontexte. Prinzipiell sind exakte Modelle $K_o$, bzw.~$G_o$ ebenfalls Hitboxen.\\
Der Term Hitbox suggeriert die Verwendung einer Box/eines Quaders zur Approximation. Das ist wahrscheinlich historisch bedingt. Der Term Hitbox ist allerdings auch für andere Formen etabliert.\\
Die Diskrepanz zwischen Hitbox und Model $\mathcal{D}(H_o) = | H_o \cup K_o \backslash (H_o \cap K_o) |$ wirkt sich negativ auf den physikalischen Realismus aus, da so False Positives/Negatives  in z.B.~Kollisionsroutinen entstehen.\\
Eine vereinfachte Darstellung eines Modells durch eine Hitbox mag jedoch Berechnungsvorteile bieten.\\
Mehrere Hitboxen können zu einem Objekt angegeben sein $H_{o, [i]}$. Dafür können mehrere Gründe angegeben werden:
\begin{enumerate}
\item Um kompexere Hitboxen durch Komposition darzustellen $H_{o, [ges]} = \bigcup_{\forall i}H_{o, [i]}$
\item Um durch verschachtelte Hitboxen so ein Objekt räumlich sukzessiv zu approximieren.
Typischerweise soll dann eine Ordnung nach der Diskrepanz $H_{o, [i]} < H_{o, [j]} \Leftrightarrow \mathcal{D}(H_{o, [i]}) < \mathcal{D}(H_{o, [j]})$ gelten. Die Hitbox mit der geringsten Diskrepanz, die Hitbox $H_{o, [min]}=min\{H_{o, [i]}\}$, wenn modellgenau $H_{o, [min]} = K_o$, wird dann als finale Hitbox bezeichnet.
\end{enumerate}

\begin{figure*}
	\begin{subfigure}[t]{0.45\textwidth}
		\centering
		\includegraphics[width=1\textwidth]{./res/csgo_hitbox.png}
		\caption{Hitbox des Spieler-Modells aus dem Videospiel Counter Strike: Global Offensive; sichtbares Modell(links), mit eingeblendeten Hitboxen (rechts)}
%%TODO source for pic
		\label{fig:chitbox}
	\end{subfigure}
~
	\begin{subfigure}[t]{0.2\textwidth}
		\centering
		\includegraphics[width=1\textwidth]{./res/pig_hitbox.png}
		\caption{Hitbox eines NPC-Modells (Schwein) aus dem Videospiel Minecraft; Hitbox in weiß}
		\label{fig:mphitbox}
	\end{subfigure}
~
	\begin{subfigure}[t]{0.2\textwidth}
		\centering
		\includegraphics[width=1\textwidth]{./res/wither_hitbox.png}
		\caption{Hitbox eines NPC-Modells (Wither) aus dem Videospiel Minecraft; Hitbox in weiß}
		\label{fig:mwhitbox}
	\end{subfigure}

	\caption{Güten von Hitboxen}
	\label{fig:hitbox}
\end{figure*}

Die Abbildungen~\ref{fig:hitbox} zeigen Hitboxen in 2 verschiedenen Spielen die die kreativen Freiheitsgrade bei der Wahl von Hitboxen und ihrer Diskrepanzen verdeutlichen sollen.\\
\ref{fig:chitbox} zeigt das Spielermodell aus dem Spiel Counter-Strike: Global Offensive (CSGO). Die angezeigten Hitboxen sind hier Ellipsoide.
Die Partition in einzelne Hitboxen für ein Modell ist ebenfalls zu erkennen.
Einzelne Details des Spielermodells, wie Riemen und Taschen an der Ausrüsung, sind im Spiel nicht essentiell und werden daher auch physikalisch nicht abgebildet.\\
Die Partition der Hitboxen in CSGO ergibt sich direkt aus einer Anforderung der Anwendung, Schusstreffer auf verschiedene Teile des Spielermodells unterschiedlich zu bewerten. Beispielsweise verursacht der Treffer am Kopf am meisten Schaden. CSGO modelliert die unterschiedlichen Treffbaren teile des Modells also über mehrere Hitboxen.\\
CSGO ist ein Shooter. Schnelle Reaktion und genaues Zielen sind ein Hauptbestandteil des Produkts. Zudem ist CSGO ein hoch kompetitiver E-Sport, der professionell gespielt wird. Es geht dabei um Preisgelder im siebenstelligen Bereich \cite{csgoprice}. Akkurate und, aus der perspektive des Spielers deterministische Hitboxen sind daher essenziell für das Produkt.\\
Abbildungen~\ref{fig:mphitbox} und~\ref{fig:mwhitbox} zeigen Hitboxen aus dem Spiel Minecraft bei zwei Nicht-Spieler-Charakteren. Die hohe Diskrepanz ist merklich. Mehr noch: Die Minecraft-Hitboxen sind koordinatenachsenparallel, d.h. Kanten verlaufen immer entlang der Koordinatenachsen der Raumrepräsentation und drehen sich nicht bei der Drehung des Modells.\\
Minecraft ist ein Sandbox Aufbauspiel. Ziel des Spiels ist der Bau von beliebigen Gebäuden, Tunneln, die Kreation von Maschinen, das Erkunden von Gebieten und was dem Spieler sonst noch einfällt.\\
In Minecraft steckt auch eine erhebliche Summe Geld. Am 15. September 2014 kaufte Microsoft die Entwicklerfirma und die rechte am Spiel für ca. 2,5 Milliarden Dollar \cite{buyminecraft}.\\
Minecraft ist nicht mit Fokus für schnelle Spieler-gegen-Spieler Szenarien kreiert. Die gesamte Spielwelt ist aus sichtbaren achsenparallel aufgestellten Würfeln aufgebaut, welche durch achsenparallele Hitboxen perfekt abgebildet werden können. Minecraft macht es sich weiter offenbar einfach und verwendet diese an Modellen wieder. Tatsächlich werden künstlich kleinere Hitboxen manchmal sogar eingesetzt um einen Treffer künstlich zu erschweren (vgl. Abbildung \ref{fig:mwhitbox}).\\
Vielleicht mag eine indirekt proportionale Korrelation zwischen der Diskrepanz bei Hitboxen und Erfolg bestehen, jedoch kann die Kritikalität von Diskrepanz nicht in allen Fällen bestätigt werden.

\subsection{Bounding-Volume}
Ein Bounding-Volume zu einem Objekt $o$ ist eine kompakte Menge $B_o \supseteq K_{o}$. $B_o$ kann als Hitbox fungieren.\\
Eine Bounding-Box ist ein spezielles Bounding-Volume in Form eines Quaders.\\
Eine in diesem Projekt extensiv verwendete, tiefere Spezialform der Bounding-Box ist die Axis-Alligned-Bounding-Box (AABB). Alle Kanten dieser Bounding-Box sind achsenparallel zu den Koordinatenachsen $\{(1,0,0), (0,1,0), (0,0,1)\}$ des 3D-Koordinatensystems.\\
Hier relevante Eigenschaften dieser sind:
\begin{itemize}
\item kleine Datenrepresentation:
		$$AABB_o \in \mathcal{S}^{3^2}; AABB_o = (BB_{min}, BB_{max}) = ((x_{min}, y_{min}, z_{min}), (x_{max}, y_{max}, z_{max}))$$.
		 In ihnen werden Minimal- und Maximalpositionen der AABB festgehalten.
		 Diese Positionen werden im Worldspace $\mathcal{S}^{3^2}$ angegeben, da AABBs hier die primäre vereinfachende Abstraktion sein sollen, die in der Simulation für Objekte verwendet wird. Die Angabe im Worldspace macht AABBs für Berechnungen im absoluten, zum Beispiel räumliche Partitionierung für Teile-und-Herrsche Algorithmen, direkt zugänglich.
		 Die theoretische, kompakte Punktemenge 
		 \begin{align*}
		 \mathcal{AABB}: \mathcal{S}^{3^2} \mapsto \mathcal{F}^3;
		 \mathcal{AABB} ((x_{min}, y_{min}, z_{min}), (x_{max}, y_{max}, z_{max})) = \\
		 \{meter((x_{min} + x * (x_{max} - x_{min}), y_{min} + y * (y_{max} - y_{min}),\\
		  z_{min} + z * (z_{max} - z_{min}))| x, y, z \in [0,1] \} 
		 \end{align*}
		 ist dann ein Bounding-Volume $\mathcal{AABB}(AABB_o) \supseteq K_o$
	\item Ermittlung einer minimalen AABB für ein Objekt durch Suche der Minima und Maxima der Ausdehnung eines Objektes in jeder Koordinatenachse:
	 $x_{min} = x : (x, y, z) \in V_o , \forall (x', y', z') \in V_o: x \leq x'; y \& z, min \& max $ analog.
	 Die Findung dieser Werte ist in $ \mathcal{O}(|V_o|) $.
		Bei rotierenden Objekten ist auch eine minimale AABB zu diesem Objekt einer Bewegung ausgesetzt, standardmäßig durch Neuermittlung der AABB($\mathcal{O}(n)$ pro Tick). Optimierungen für verschiedene Arten von Bewegung möglich (Positionsänderung, Skalierung, etc.), aber manchmal schwierig bis unmöglich (z.B. bei Rotation).\\
		Aus diesem Grund macht es auch Sinn, nicht-minimale AABBs zu wählen.
	\item Schnelle Kollisionsüberprüfung zwischen AABBs durch Vergleiche der Extrema $\mathcal{O}(1)$
\end{itemize}

AABBs, bzw. Bounding-Volumes generell, werden nicht ausschließlich für statische Objekte $K_{o,t}$ zu einem Zeitpunkt erstellt und verwendet. Es macht in bestimmten Kontexten zum Beispiel auch Sinn das komplette durchlaufene Volumen eines Objektes $\{K_{o,t'} | t' \in \Upsilon_{\delta_i}\}$ während eines Ticks durch ein Bounding-Volume zu abstrahieren.

Im Falle des Spiels Minecraft werden AABBs als finale Hitboxen verwendet (vgl. Abbildungen \ref{fig:mwhitbox}, \ref{fig:mphitbox}), welche jedoch scheinbar dem Kriterium $K_o \subseteq B_o$ zuwiderlaufen. Es muss an dieser Stelle zwischen der mathematischen Korrektheit einer Bounding-Box gegenüber einem gegebenen physikalischen Modell und der Designentscheidung gemacht werden, dass das sichtbare Modell nicht oder nur marginal die Grundlage des physikalischen Modells ist. In Minecraft ist die AABB die finale Hitbox  $H_{o, [min]} = K_o$ und definiert dadurch selbst das Modell $K_o$. Das Bounding-Box-Kriterium ist damit theoretisch erfüllt. Die Designentscheidung selbst soll an dieser Stelle nicht eingeschätzt werden.



\section{Hergang}
Der Entwicklungsprozess in diesem Projekt umfasst das Erreichen bestimmter Meilensteine, die konstruktiv auf das Endziel hinführen. Dieser Vorgang wird gewählt, um die Gesamtkomplexität des Endziels zu abzuschwächen und Bearbeitungszeiten einzuschätzen. Durch die Abhängigkeit, die das Projekt vom Status quo des Vorprojektes hat, kann so dieses sukzessive an die hier gestellten Anforderungen angepasst werden.\\
Der Hergang lautet wie folgt:
\begin{enumerate}
\item Aufarbeitung des Status Quo\\
Umsetzung eines Designs im bestehenden Basisprojekt, das mit weiteren in diesem Projekt zu entwickelnden Features kompatibel ist.
\item Implementierung einer Fernbedienung\\
Fernbedienung als erstes netzwerkabhängiges Feature mit Einwegkommunikation, die keine Synchronisation benötigt.
\item Erneute Aufarbeitung
\item Implementierung der Echtzeitsynchronisation zwischen Simulationen\\
Auf zwei über Netzwerk verbundenen Maschinen wird jeweils eine Simulationsinstanz ausgeführt, deren Inhalte Synchronisiert werden.
\end{enumerate}



\section{Aufarbeitung des Status Quo/Vorbereitung zur Fernsteuerung}
Die Fernsteuerung ist als Steuerung zu verstehen, d.h. keine Rückläufige Information aus der Simulation wird benötigt. Die von einem Sender gelieferte Information wird nur gelesen/konsumiert und entsprechend reagiert, es erfolgt aber keine Antwort. Die Fernsteuerung wird also als klassische Einwegkommunikation verstanden.\\

Zum Zwecke der zukünftigen Umsetzung des Features der Fernsteuerung soll zunächst einmal der Status Quo in seinen hier relevanten Teilen Dargestellt werden. 

\subsubsection{Abstraktionen}
Dazu werden zunächst Terminologien für Abstraktionen erklärt. Nicht alle diese Abstraktionen sind auch exakt so im Code vorhanden, aber ihr Verwendungszweck ist prinzipiell umgesetzt. 



\begin{enumerate}
\item Simulation\\
Die Simulation ist der Backend-Kern des Programms. Dieser Teil kümmert sich um die korrekte Umsetzung von Inhalten der Simulation und ihrer Ausführung. Die Abstraktion dient auch der Möglichkeit der Umsetzung verschiedener Objektive der Simulation.
\item Operator\\
Der Operator ist die Repräsentation eines (menschlichen) Bedieners und regelt somit bestimmte Ein- und Ausgabevektoren der Simulation.
\item Operatorentität\\
Eine spezielle Entität, welche als Avatar für den Operator in der simulierten Welt dient. Diese Spezialisierung ist essentiell im Bereich von Videospielen, um dem Spieler Zugang zur simulierten Welt zu geben. Die Operatorentität implementiert Reaktionen in Form von Aktionen der Entität auf bestimmte Eingabeparameter und ermöglicht den Erhalt einer Perspektive, v.A.~für Zwecke des Renderings.
\item Ereignisverarbeitung\\
Über Bibliotheken werden Betriebssystemressourcen angefordert, wie z.B. ein Fenster. Ein Fenster ist in der Lage, Eingaben von Eingabegeräten (Tastatur, Maus, etc.) und Fenster bezogene Information (Größenänderung, Status des Fokus, etc.) in Form sog.~Ereignisse(eng. \textit{events}) zu liefern. Verhalten des Fensters sind größtenteils von den verwendeten Bibliotheken gekapselt, jedoch erfordern bestimmte Aspekte eine Reaktion weiterer Komponenten (z.B. beim Schließen des Fensters sollen Ressourcen freigegeben werden, Simulation geschlossen werden, etc.).
\item 3D Renderer der Simulation\\
Die graphische Ausgabe der Simulation ist prinzipiell keine Aufgabe der Simulation selbst, die theoretisch auch ohne Ausgabe korrekt arbeiten müsste. Ein Renderer greift lesend in viele Aspekte der Simulation ein, um die Informationen für seine Aufgabe zu erhalten.

\item Applikationskopf\\
Als zentrale Organisationsstruktur des Programms ist es die Aufgabe des Applikationskopfes, den Programmablauf zu Regeln. Das Programm besteht in seinem Kern aus einer Schleife, in der wiederholt andere genannte Subkomponenten ihre Verarbeitungsschritte durchführen. Der Applikationskopf dient dabei als Abstraktion für das Verhalten der Gesamtapplikation. So können verschiedene Verwendungszwecke, bzw. Arten der Gesamtapplikation (Standardprogramm, Fernbedienung, Simulation mit Fernbedienungsempfänger, etc.) realisiert werden. In der Zukunft prinzipiell Versionen mit Multi-Threading der verschiedenen Subvorgänge denkbar.

\item OS-Zugang\\
Enthält Zugriff und Schnittstellen zu Ressourcen des Betriebssystems, dem Fenster, Ereignisquelle, Lautsprecher, etc.
\end{enumerate}


\subsubsection{Status-Quo Design}
Durch die Betrachtung der Abhängigkeiten und Positionen im Design kann ein neues Design veranschlagt werden, das kompatibel mit zukünftigen Änderungen ist, und aus der Diskrepanz zwischen den Designs ein Änderungsplan erstellt werden.\\

\begin{figure}

\centering
\resizebox{.9\linewidth}{!}{
\begin{tikzpicture}[thick,scale=1, every node/.style={scale=1}]
\begin{package}{Applikation}
\begin{class}{Applikationskopf}{0,0}
\end{class}
\begin{class}{Simulation}{9,-2}
\end{class}
\begin{class}{Ereignisverarbeitung}{3,-2}
\end{class}
\begin{class}{OS/Window}{-3,-2}
\end{class}
\begin{class}{Entity}{9, -4}
\end{class}
\begin{class}{Operatorentity}{3,-4}
\inherit{Entity}
\end{class}


\aggregation{Applikationskopf}{1}{}{Simulation}
\composition{Applikationskopf}{1}{}{OS/Window}
\composition{Applikationskopf}{1}{}{Ereignisverarbeitung}
\association{Ereignisverarbeitung}{1}{}{Operatorentity}{1}{}
\association{Simulation}{1}{}{Operatorentity}{1}{}
\association{OS/Window}{1}{}{Ereignisverarbeitung}{1}{}
\aggregation{Simulation}{1..*}{}{Entity}

\end{package}
\end{tikzpicture}
}
\caption{UML-Perspektive auf den Status-Quo zu Beginn des Projekts}
\label{fig:status_quo}
\end{figure}

Die Abbildung \ref{fig:status_quo} zeigt eine Darstellung von in etwa dem Status Quo zu Beginn des Projektes mit Abstraktionen und Abhängigkeiten.
In dieser Ansicht werden schnell einige Impraktikabilitäten, bzw. schlechte Designaspekte erkenntlich.
Das Projekt ist durch viele individualistische Entwicklungsabschnitte mit variabler Seriosität in einem Zustand von nicht intelligentem Design. 
In der Vergangenheit wurden oft kurze Implementationswege gewählt, um ein bestimmtes Feature umzusetzen. Einige dieser kurzen Wege müssen daher einem Design unterzogen werden, um die Features für dieses Projekt zur Änderung zu öffnen, möglichst ohne die nicht unbeachtliche Menge anderer funktionierender Features zu beeinträchtigen.\\

Die isolierte Betrachtung einer bestimmten Abhängigkeit mag hochabstrakt teils als wenig sinnvoll erscheinen, jedoch sind bestimmte Abhängigkeiten auch nur wenig intuitiv. So ist z.B. die Abhängigkeit des Renderers zu der Systemressource des Fensters noch trivial zu erkennen, die Abhängigkeit der Simulation vom Fenster jedoch nicht. Hier ist die Rendering-Kamera, die die Perspektive auf den Simulationsinhalt abstrahiert, eine Ressource, die einer Operatorentität gehört, und somit implizit Eigentum der Simulation ist. Die Rendering-Kamera benötigt Informationen über die Größe der Rendering-Ziels in Pixel, um bestimmte Konfigurationen vorzunehmen. Diese Abhängigkeitskette wird nicht als trivial angesehen. Um solche Ketten zu sprengen, werden u.U.~weitere Abstraktionen benötigt, die dann keinen kurzen Implementierungsweg mehr darstellen, weswegen sie nicht im ursprünglichen Design enthalten sind.\\

Man kann an der Abbildung \ref{fig:status_quo} auch beobachten, dass viele der Abhängigkeitsbeziehungen $1:1$ sind. So kann hier beispielsweise nur jeweils eine Operatorentität von einer Simulation behandelt werden.  Weiter wird hier genau eine Operatorentität gefordert, d.h. eine Simulation kann ohne einen Bediener nicht laufen. Das sind Umstände, die in Anbetracht des Ziels von mehreren aktiven Operatoren revidiert werden müssen.\\
Von Abbildung \ref{fig:status_quo} abwesend sind Komponenten zugehörig zum Rendering. Die Renderingkomponente assoziiert zum Status-Quo mit allen in der Abbildung gezeigten Komponenten, meist in $1:1$ Beziehung. Es kann also festgehalten werden, dass derzeit genau eine Gruppierung von Fenster, Operatorentität, Simulation, etc.~auf einer Maschine  auch eine grafische Ausgabe erzeugen kann. Ob dieses sehr eingebettete Feature im Rahmen dieses Projektes verändert werden muss ist zu diesem Zeitpunkt noch nicht klar. Das Feature hat Defizite im Design, die allerdings, falls irrelevant, hier ignoriert werden.


\subsubsection{Ermittlung eines Designs für die Fernsteuerung}

\begin{figure}

\centering
\begin{subfigure}[b]{0.3\textwidth}
\centering
\resizebox{\linewidth}{!}{
\begin{tikzpicture}[framed, thick,scale=1, every node/.style={scale=1}]

\begin{class}{Simulation}{0,-2}
\end{class}
\begin{class}{Operator}{0,0}
\end{class}
\begin{class}{LocalOperator}{0,2}
\inherit{Operator}
\end{class}

\unidirectionalAssociation{Operator}{1}{}{Simulation}{1}{}

\end{tikzpicture}
}
\caption{UML-Diagramm zur Umsetzung eines Lokalen Anwenders der Simulation}
\label{fig:local}

\end{subfigure}
\begin{subfigure}[b]{0.3\textwidth}
\centering
\resizebox{\linewidth}{!}{
\begin{tikzpicture}[framed, thick,scale=1, every node/.style={scale=1}]

\begin{class}{Simulation}{0,-2}
\end{class}
\begin{class}{Operator}{0,0}
\end{class}
\begin{class}{RemoteControlOperator}{0,2}
\inherit{Operator}
\end{class}
\begin{class}{RemoteControlSender}{0,4}
\end{class}

\unidirectionalAssociation{Operator}{1}{}{Simulation}{1}{}
\unidirectionalAssociation{RemoteControlSender}{1}{}{RemoteControlOperator}{1}{}

\end{tikzpicture}
}
\caption{UML-Diagramm zur Umsetzung des fernsteuernden Anwenders}
\label{fig:remotecontrol_indiv}
\end{subfigure}
\label{fig:remotecontrol_design}
\end{figure}

In Abbildung \ref{fig:status_quo} kann auf die Bidirektionalität der Assoziation zwischen Ereignisverarbeitung und Operatorentität hingewiesen werden. Die Assoziation kann hier als der Kommunikationsweg zwischen Operator und Simulation angesehen werden. Die Bidirektionalität steht hier direkt gegen die erwünschte unidirektionale Kommunikation, die für die Fernsteuerung veranschlagt wird (vgl. Abbildung~\ref{fig:remotecontrol_design}).



Unter den beschriebenen Abstraktionen wird die Umsetzung der Fernsteuerung individualisiert strukturell in Abbildung \ref{fig:remotecontrol_indiv} gezeigt. 
Zu sehen sind die relevanten Programminhalte relativ zu beiden kommunizierenden Maschinen.
Der Empfänger führt dabei die Simulation aus. Der Operator wird dabei durch eine Netzwerkverbindung geteilt.
Der Sender enthält dabei nur einen Teil des Operators, der für die Annahme von Eingaben zuständig ist, verarbeitet diese und sendet diese an den Empfänger.
Der Empfänger hostet die Simulation und einen entsprechenden Operator, welcher für die Annahme von Netzwerkkommunikation implementiert ist.
Zur Veranschaulichung der Belange der Teilprogramme Sender und Empfänger, bzw. Host und Server, werden die anderen genannten Abstraktionen im Diagramm \ref{fig:remotecontrol} eingeblendet.
Sofort sind die Unterschiede zwischen diesem Vorschlag und dem Status-Quo sichtbar.
Übergreifend lässt sich die Änderung als Extraktion der Operators aus der Simulation beschreiben. Die Schwierigkeit, bzw. der Umfang der Aufgabe, besteht darin, alte Abhängigkeiten von Komponenten umzumodellieren.\\

\subsubsection{Optionen der Übermittlung}
Verschiedene Fragestellungen über die Übertragungseigenschaften müssen beantwortet werden, um aus den verfügbaren Optionen zur Kommunikation auszuwählen.\\

\begin{enumerate}
\item Welche Echtzeitanforderungen bestehen?
\item Was wird übertragen?
\item Welche Übertragungsrobustheit ist für ein bestimmtes Datum gefordert?
\item Protokolle TCP oder UDP?
\end{enumerate}

Prinzipiell ist die ungefilterte Übertragung von Eingabeereignissen möglich und am einfachsten Umzusetzen. Daraus entstehen allerdings einige Nachteile:
\begin{enumerate}
\item Übermittlung für den Empfänger irrelevanter Information, z.B. das Drücken einer ungebundenen Taste am Sender.
\item Bündelung in Pakete durch Protokolle führt zur Bündelung redundanter Information und Verlust der relativen Auftrittszeit zwischen Ereignissen. Standardmäßig führen Ereignisse keinen Zeitstempel mit.
\item Oft viele gleichartige Ereignisse hintereinander. Frequenz von Ereignissen ist Abhängig durch Abtastraten der Verwendeten Bibliothek, Treibersoftware und Hardware der Eingabegeräte. Sogar schon im Status-Quo zu Beginn des Projektes werden Ereignisse, die keine Änderung darstellen, bereits gefiltert. Trotzdem werden z.B. beim Bewegen der Maus zehnfach Ereignisse erzeugt, welche die Mausposition nur minimal aktualisieren.
\end{enumerate}
Es erscheint daher profitabler, Ereignisinformation in einer abstrakten Darstellung der Intention eines Bedieners zu akkumulieren und diese Darstellung dann zyklisch zu übertragen.\\
Prinzipiell kann jedes kommunizierte Informationspaket als eine Aktualisierung des auf dem Empfänger bekannten Status angesehen werden.
Um eine Bedienung über Netzwerk annähernd in ihren qualitativen Eigenschaften zu einer lokalen Bedienung ohne Netzwerkverbindung herzustellen muss mehrfach pro Sekunde übertragen werden. Wie oft muss ein entsprechender Test zeigen. Die Schätzung an dieser Stelle liegt bei ca.~20Hz.
Über Netzwerkverbindungen existieren Informationsverluste, z.B. durch Paketverlust. Das Netzwerkprotokoll TCP mitigiert Paketverlust typischerweise völlig, ist dadurch aber nicht mehr Echtzeitfähig. TCP-Kommunikation blockt bei Paketverlust, bis das verlorene Paket nachgesendet worden ist. 
Ein Beispiel der unvorteilhaften Verwendung von TCP im Bereich der Videospiele ist das äußerst erfolgreiche und bekannte Spiel Minecraft von Mojang. Hier äußern sich Netzwerkverzögerungen (Lags) durch einen Stillstand der Spielabläufe, der sich über Sekunden ziehen kann und anschließendes Aufholen der Simulationszeit im Zeitraffer. Manch ein unverdientes Game Over kann daher von einem Gegner verursacht werden, der in der Zeit des Stillstandes Angriffe auf den Spieler im TCP-Puffer ansammelt, auf die der Spieler nicht reagieren kann.\\

Bei der Verwendung von UDP muss allerdings Übertragungsrobustheit selbst hergestellt werden. \\
Klassisch wird dafür oft Redundanz als stochastischer Lösungsansatz vorgeschlagen/gelehrt.
Weiter steht UDP TCP in puncto Empfangsbestätigung nach. Für die hier veranschlagte unidirektionale Kommunikation der Fernsteuerung ist die Rücksendung einer Empfangsbestätigung allerdings nicht akzeptabel.\\
Es wird daher ein Trick verwendet:
Beim zyklischen Versand kann explizite Redundanz (also das mehrfache Senden einer Nachricht) in eine implizite Variante umgewandelt werden, wenn in jedem Zyklus ein Gesamtstatus übertragen wird.
Dieser Ansatz ist für kleine Gesamtgrößen eines Status noch tragbar.
Es werden in einem Paket also absolut Daten versendet anstatt eine relative Änderung. Das Prinzip gleicht dabei dem eines anliegenden Pegels. Beispiel:
$x = 16$ Wir senden nicht \glqq Addiere $+32$ auf Variable $x$\grqq , sondern \glqq Variable $x = 48$\grqq .\\
Diese Art vorzugehen kann auch auf zu konsumierende Signale angewendet werden, indem anstatt ein Signalbit ein Zählerstand übertragen wird, der die Anzahl der übertragenen Signale zählt. Ein Empfänger kann so bei Paketverlust trotzdem die verlorene Information aus der Differenz zum vorigen Zählerstand ablesen.
%TODO graphic
Eine besonders nützliche Eigenschaft der zyklischen Übertragung kann in diesem die Assoziativität der Aktualisierung sein. Wir definieren die Aktualisierungsoperation $\oplus : STATUS \times STATUS \mapsto STATUS$. Wir gehen davon aus, das ein Zeitstempel (Zeit definiert am Sender) der Erzeugung des Status in diesem enthalten ist $timestamp: STATUS \mapsto T_r$.
$$ status, aktualisierung \in STATUS : status \oplus aktualisierung = s' $$
$$
s' =
\begin{cases}
	aktualisierung,& \text{wenn } timestamp(aktualisierung) > timestamp(status)\\
    status,              & \text{sonst}
\end{cases}
 $$
 Dadurch gilt Assoziativität von $\oplus$ : $ s_1 \oplus (s_0 \oplus s_2) = (s_1 \oplus s_0) \oplus s_2 $
 Wenn $timestamp(s_0) < timestamp(s_1) < timestamp(s_2)$, dann ist nach dem Empfangen von $s_2 s_1$ irrelevant. Unter den zuvor beschriebenen Eigenschaften ist die überschreibend zyklische Übertragungsmethode vorteilhaft.\\
Abstrakte Bedienerintentionen werden also in einem absolutem, zyklisch übertragenen Status dargestellt.\\
Beispiele:
\begin{enumerate}
\item Absolute Rotation (Ausrichtung, nicht Winkelbeschleunigung) der Operatorentität als 3D-Vektor.
\item Waffenselektion als Index (Integer)
\item Neustartsignal der Simulation als Signal ( Signalzähler, Integer )
\end{enumerate}
In einem Shooter-Videospiel muss geschossen werden können. Mit den nun bekannten Einschränkungen entstehen dafür jedoch einige Probleme:
\begin{enumerate}
\item Zwischen Sendevorgängen kann ein Auslöser mehrmals betätigt werden.
\item Die zeitliche Abstimmung dieser Eingabe ist relevant mit strengen Echtzeitbedingungen, denn zum Beispiel 200ms nach dem Betätigen der entsprechen Taste zum Schießen kann das Ziel bereits getroffen sein müssen.
\item Das Verhalten einer Waffe beim Drücken des Auslösers variiert je nach Waffe, ist also in der Entität auf dem Empfänger definiert. Es ist also nicht möglich, lokal auf dem Sender, der keine Information über die konkrete Waffe besitzt, abstraktere Information zu ermitteln und das Feature so umzusetzen.
\end{enumerate}
Man könnte theoretisch Zeitintervalle übertragen, um die Auslösezeiten auf dem Empfänger rückwirkend herzustellen aber ohne eine Synchronisation, d.h. bidirektionale Kommunikation, der Zeiten zwischen Sender und Empfänger kann das erwünschte Verhalten nicht exakt rekonstruiert werden. Zudem ist rückwirkender Einfluss auf physikalische Zustände in einer Simulation schwierig Umzusetzen, wenn echtzeitr++elevante Information zu spät eintrifft.
Um den Fall der mehrfachen Betätigung abzudecken, müsste eine ganze Menge solcher Intervalle übertragen werden. Selbst wenn vollständig rekonstruierbare Übertragung möglich wäre, sind die Anforderungen auf die Reaktionszeiten dieses Features so streng, dass nur leichte Verzögerungen/Latenzzeiten (ca. 10ms) schon einen Bruch dieser Anforderungen bedeuten.\\
Diese Konzepte sind im Rahmen der Fernsteuerung nicht zufriedenstellend verlustfrei und es müssen Abstriche gemacht werden. Das Feature wird zunächst durch einen übertragenen Boolean dargestellt, der trivial den Zustand des Auslösers darstellt, und die bei hohen Latenzzeiten negativen Eigenschaften der dieser Entscheidung in Kauf genommen.\\
Dies geschieht vor allem spekulierend auf die später im Projekt einzufügende Echtzeitsynchronisation, unter deren Umständen dieses Feature besser umgesetzt werden kann.\\
Eine hier unwesentlich bessere Umsetzung selbst unter hier aktuellen Umständen ist denkbar, soll an dieser Stelle jedoch ignoriert werden.


\section{Problemdefinition}
Der Begriff Kollisionserkennung (\textit{eng. collision detection}) scheint, je nach Anwendungsfall, ein spezifisches Problem zu sein und beschreibt eine Sammlung verschiedener Teilprobleme.\\
Wir beschränken uns für die Zwecke dieses Projektes auf Folgende:
\begin{enumerate}
\item Die Ermittlung der Menge von Objektpaaren $C \subseteq OBJ^2$ aus einer Menge von sich bewegenden Objekten $OBJ$ in einem Raum, welche sich innerhalb eines Zeitraums, hier einem Tick $\delta_i$, überschneiden.
$$C = \{(o_0, o_1) | o_0, o_1 \in OBJ; t\in \Upsilon_{\delta_i}; G_{o_0, t} \cap G_{o_1, t} \neq \emptyset\}$$
\item Die Ermittlung von Informationen über den Hergang einer Kollision.
Die genauen Anforderungen, welche Informationen ermittelt werden müssen, gehen aus den Verwendungszwecken hervor (vgl.~\ref{sec:usages}). 
\end{enumerate}

Während mögliche Anforderungen von Verwendungszwecken zwar betrachtet werden müssen, sollen die Verwendungszwecke selbst in diesem Projekt konkret nicht realisiert werden.

Wir schätzen an dieser Stelle modellgenaue Kollisionen als schwierig ein. In Teilproblem 1 entsteht daraus das Problem einer schnell steigenden Komplexität, da die Anzahl der möglichen Kollisionspaare quadratisch wächst $|C|\leq |OBJ|^2$.\\
Das Teilproblem 1 wird daher in zwei separaten Schritten behandelt.
\begin{itemize}
\item[1.1] Elegante und effiziente Vorfilterung von Objekten $OBJ$, um die modellgenaue Überprüfung von großen Mengen von Objektpaaren in $C$ zu vermeiden.
\item[1.2] Die abstrakte Behandlung des modellgenauen Kollisionsproblems bei genau zwei Objekten. Dabei werden Algorithmen gewählt, welche die für Teilproblem 2 benötigten Informationen mitliefern.
\end{itemize}

Für beide dieser Schritte sollen im Folgenden Algorithmen beschrieben werden.\\
\\
\label{sec:physical_realism}
Es ist weiter üblicherweise ein Ziel von physikalischen Simulationen einen realen Sachverhalt möglichst akkurat und konform mit dem etablierten physikalischen Verständnis umzusetzen. Diese Anforderungen müssen für den bestehenden Verwendungszweck der Simulation relativiert werden.\\
In Videospielen dient die akkurate Umsetzung nur der Immersion des Konsumenten,~d.h. der Konsument soll der Illusion von absolutem Realismus ausgesetzt sein, während absoluter Realismus aber schwer bis gar nicht umsetzbar sein kann. Je nach Umstand ist sogar der Bruch der Illusion in Maßen nicht kritisch. Es sollte dem Konsumenten möglich sein mit Hilfe seines vorhandenen physikalischen Verständnisses ein Verständnis für die Vorgänge der Simulation intuitiv entwickeln zu können.\\
Videospiele sind im Kontext der Simulationsgenauigkeit generell sehr vergebend. Zum Vergleich: Simulationen mit zu niedrigen Fehlertoleranzen in z.B.~der Industrierobotik können zu erheblichem Sach- oder gar Personenschaden führen.
Dem Entwickler ist die Aufgabe gestellt, zwischen den Faktoren Performanz, Realismus und Umsetzbarkeit einen Vernünftigen Kompromiss zu finden. Die Unterschiedlichkeit dieser Kompromissfindung kann in~\ref{sec:hitbox} am Beispiel von Minecraft und CSGO betrachtet werden. Realismus ist dort kein ausschlaggebender Faktor für Erfolg.\\

Auch in diesem Projekt müssen die Erwartungen an physikalischen Realismus relativiert werden. Es bestehen Freiheitsgrade in sowohl der Definition der physikalischen Vorgänge als auch der algorithmischen Umsetzung. 
Videospiele, als Kunstform, können so auch surreale Konzepte umsetzen und Echzeitanforderungen können leichter eingehalten werden.


\section{Zusammenfassung}
Da die Arbeit an der Codebasis nicht auf das HSP-Projekt beschränkt ist, ist eine zukünftige Arbeit an einer Verbesserung der Synchronisation wahrscheinlich. Der aktuelle Stand weist noch einige Probleme und einiges an Optimierungspotenzial auf:
\begin{itemize}
\item Die Sichtrichtung einer Spielerfigur, und somit die grafisch angezeigte Perspektive am Client, wird aktuell noch vom Server bestimmt, ohne dass der Client für die Kamera im 3D-Raum eine eigene Schätzung verwendet. 
Ein Spieler muss so beim Drehen seiner Sicht die Zwei-Wege-Latenz zum Server in Kauf nehmen. Zwar handelt der Server nach dem hier gezeigten Synchronisationsschema mit Gleichzeitigkeit, der Unterschied wird jedoch am Client grafisch merkbar, wenn sich die Perspektive schnell ändert während z.B.~geschossen wird. Projektile sind prinzipiell da wo sie sein sollen, das lokal angezeigte Fadenkreuz u.U.~allerdings nicht. Für einen Shooter ist die Flüssigkeit des Bildes  beim Zielen essentiell.
Dies wäre somit die höchste Priorität für Erweiterungen des klassischen linearen Extrapolationsmodells der Grafik am Client, die lokal anliegenden Eingabeereignisse eigenständig mit einzubeziehen. Problematisch bei der Umsezung ist dort, dass der Server bestimmt, wann der Spieler seine Waffe abfeuert. Er bestimmt damit auch den Rückstoß, der die Blickrichtung ändert.
\item Geräusche, die alle als Reaktion auf ein Ereignis im Spiel auftreten (also vom Server angeordnet werden), wurden im Rahmen des Projekts nicht synchronisiert. Der Grund dafür ist, das aufgrund der Art und Weise, wie diese bisher implementiert sind und eine weitgehende Neuimplementierung des Features notwendig wäre, um auf dem Client zu funktionieren. Das wurde für den Rahmen des Projekts als zu Aufwändig eingeschätzt.
\item Informationen sind aktuell nicht priorisiert. Es wäre jedoch für die gefühlte Latenz von Vorteil, wenn Updates z.B.~zur eigenen Spielfigur in jedem Paket enthalten wären, und nicht nur, wenn diese im Round-Robin Verfahren an der Reihe is. Da aber aktuell der Syncable-Manager nur ein einziges Paket erzeugt, das von der Netzwerkkomponente an alle Clients gesendet wird, muss dort eine signifikante Anpassung erfolgen, um für unterschiedliche Clients unterschiedliche Updates zusammenzustellen. Über ein ähnliches System könnte man außerdem die Genauigkeit von Update-Paketen auf den konkreten Client zuschneiden. So wäre es z.B. vertretbar, bei einem weit entfernten Gegner dessen Sichtrichtung nur mit einer Genauigkeit von 1\,Byte pro Achse zu übertragen.
\item Die bisherige Implementierung erlaubt neben einzelnen Kreierungen und Löschungen von Objekten auch die Einbindung beliebiger Events. Über diese könnte man bisherige bandbreitenintensive Kreierungsbefehle zu Gruppen zusammenfassen, die z.B. mit einem deterministischen Pseudozufallsgenerator auf beiden Maschinen mit minimalem Datenaustausch identische Resultate liefern. Man könnte diese Events außerdem benutzen, um die Updates für eine große Menge an Entitäten zu ersetzen. Dies wäre möglich, indem diese in einer Gruppe zusammengefasst wären und intern deterministisch agieren (z.B Gegner marschieren in Formation). Die übertragenen Events wären dann Befehle, die deterministisch die Aktionen über einen bestimmten Zeitraum bestimmen.
\item Der Client könnte komplexere Vorausberechnungen als die lineare Extrapolation anstellen. So könnte er z.B. temporäre Entitäten erzeugen, die nur solange existieren, bis Serverinformationen über deren Erstellungszeitpunkt den Client erreicht haben. Sie würden dann durch die aktuellen Informationen vom Server ersetzt werden (oder ersatzlos gestrichen, wenn der Server für diesem Zeitpunkt keine Erzeugung einer Entität vorgesehen hat). So könnte der Spieler z.B. lokal ein abgeschossenes Projektil sofort sehen, nicht erst wenn es vom Server bestätigt wurde.
\end{itemize}

\newpage
\printbibliography


%\bibliographystyle{hieeetr}
%\addcontentsline{toc}{chapter}{Bibliography}
%\bibliography{grr.bib}
%\begin{thebibliography}{12}
%\bibitem[HaKT1 98]{HaKT1 98} \footnote{In die 
%Bibliographie sollte s"amtliche benutzte Literatur 
%rein, auch nicht beim eigenen Vortrag angegebene, aber benutzte Papiere 
%und B\"ucher. Gleichzeitig sollte aber alles in der Literaturliste angegebene
%mindestens einmal im Artikel zitiert werden, sonst nicht auflisten.}
        %Michael Harkavy, J. D. Tygar, Hiroaki Kikuchi: {\sl Multi-round 
        %Anonymous Auction Protocols}; 1st IEEE Workshop on Dependable and 
        %Real-Time E-Commerce Systems, 1998.

%%Here go sources, which both grr and hel might need
% for example
\bibitem[1]{cd2D}
	Jimenez-Delgado, Juan \& Segura, Rafael \& Feito, Francisco. (2004). Efficient Collision Detection between 2D Polygons.. Journal of WSCG. 12. 191-198. 



%\input{grr_bibliography.tex}
%\bibitem[TseDef\_80]{tse}
        Seitz Default: {\sl How to Cite something 2}; 
        Communications of the ACM 22/11 (1979), S. 612-613.

%\end{thebibliography}

\end{document}





