\documentclass[11pt,twoside,a4paper]{article}

\usepackage{a4wide,amsmath,amssymb}
\usepackage[school, simplified]{pgf-umlcd}
\usetikzlibrary{calc}
\usetikzlibrary{positioning}

% Mann will direkt Umlaute eingeben können statt \"a, \"o, \"u usw.
% Entweder:
\usepackage[utf8]{inputenc}
% oder:
%\usepackage{umlaut}
\usepackage[german]{babel}

\usepackage[style=numeric]{biblatex}
\addbibresource{grr.bib}

\usepackage{textcomp}
\usepackage{graphicx}
\usepackage{subcaption}

\usepackage{hyperref}

\usepackage{tikz}
\usepackage{background}
%uncomment next line to remove "Draft" watermark
%\backgroundsetup{contents={}}


% Trennvorschl"age (in {} einfuegen, wenn nicht automatisch getrennt wird:
% z.B. Authen-ti-ka-tions-sys-tem)
%\hyphenation{}

%\hyphenation{min-des-tens}
%\hyphenation{Kol-li-sions-er-ken-nung}


%-------------------------- Formatsachen --------------------------%

% Bild-, Tabellenunterschriften veraendern:
% Nummer fett, kleinerer Text fuer Bildunterschrift
%\usepackage[bf,small]{caption}


%\usepackage{mathpazo}  % -- Palatino als Zeichensatz -- einfach diese
					   % Zeile auskommentieren, falls nicht installiert
%\usepackage{mathptmx}  % -- Times als Zeichensatz

% Zum Unterscheiden von Entwurfs- und endgueltiger Fassung
%\usepackage{draftcopy}
%\draftcopySetGrey{0.90}   %   90% = sehr helles Grau
%\draftcopyName{ENTWURF}{155}   % statt ``DRAFT''
%\draftcopySetScale{1}

%--------------- Zeilen- und Absatzabstaende ----------------------%
%\setlength{\parindent}{0em}
%\setlength{\parskip}{\medskipamount}    % Abstand zwischen Abs"atzen


\begin{document}

\title{HSP-Projektarbeit im Master Informatik \\
\small Echtzeitsynchronisation von Simulationen mit Fokus auf die Verwendung in Videospielen}
\author{Robert Graf, Lukas Hermann\\
%  (\texttt{fridolw@in.tum.de})\\[5mm]
%  Seminar "`Internetrouting"' , \\
  Ostbayerische Technische Hochschule Regensburg\\
  \\
  Projektbetreuung: Prof. Dr. Markus Kucera
}
  
\date{WS\, 2019/2020 (Version vom \today)}

\maketitle

\newpage
\tableofcontents


\abstract{Das Projekt umfasst die Erstellung, bzw.~Umwandlung, einer 3D-Simulation mit dem Fokus auf die Umsetzung einer Echtzeitsynchronisierung besagter Simulation zwischen zwei Simulationsinstanzen auf verschiedenen Rechenmaschinen, bzw.~Prozessen. Dabei werden sowohl Techniken der verlustarmen Implementierung der Synchronisation erschlossen, als auch benötigte Anforderungen und Designaspekte ermittelt.}


\section{Einleitung}


\subsection{Kontext und Motivation}
Eine Echtzeitsynchronisation einer Simulation auf mehrere über Netzwerk verbundene Rechenmaschinen soll die Äquivalenz der Simulationszustände auf allen beteiligten Maschinen zum selben, oder annehmbar ähnlichen Zeitpunkt gewährleisten.\\
Das Ziel ist die Kollaborative Verwendung der Simulation. 
Dazu zählen sowohl die Anzeige von Simulationsinhalten, als auch
die benutzerdefinierte Veränderung dieser. 
Die Synchronisation der Simulationszustände beinhaltet demnach die rechtzeitige Synchronisation der Änderungen.

Nach dieser Definition oder nach dessen Teilen, finden wir Echtzeitsynchronisation in industriellen Anwendungen wieder. Zum Beispiel kann im Bereich der Robotik der rechtzeitige Informationsfluss zwischen Steuergeräten und Sensoren als Echtzeitsynchronisation angesehen werden.
%%TODO cite example
Das wohl ähnlichste Anwendungsbeispiel im Kontext dieser Arbeit lässt sich im Bereich der Videospiele, oder Bedienersimulationen finden, wo ein echtzeitsynchronisierter Simulationsstatus auf mehreren Maschinen lokal angezeigt werden muss, Bediener jedoch auch selbst Einfluss auf die Simulationsinhalte nehmen können.\\
In diesem Projekt wird auf einer bestehenden Codebasis einer Simulation, welche nicht unter Einbezug von Synchronisationsgedanken entwickelt wurde, eine solche umgesetzt. Dabei soll eine Echtzeitsynchronisation zwischen mehreren Rechenmaschinen über eine Internetverbindung mit 100\,KByte/s Bandbreite erreicht werden.
Das letztendliche Ziel ist die möglichst unmerkliche Synchronisierung zweier Simulationszustände und deren Anzeige, d.h. die Reduzierung von Seiteneffekten durch Latenz.\\
Im Laufe der Arbeit werden Kernaspekte zur Synchronisation, aber auch zum benötigten Softwaredesign ermittelt, um synchrone Applikationen umzusetzen. Durch den Status-Quo der übernommenen Codebasis werden durch die Überarbeitung dort gemachte Fehler in diesen Aspekten erkenntlich.

\subsection{Beschreibung der konkreten Simulationsanwendung}
Die verwendete Simulationsanwendung ist ein rudimentäres 3D-Shooter-Videospiel. Ein Spieler kann dabei in einem Terrain auf Gegnerfiguren schießen, diese beschädigen und wird dabei durch Punkte entlohnt. Diese einfache Anwendung hält bereits alle Kernaspekte einer Simulation im Kontext dieser Arbeit inne:
\begin{itemize}
\item Die Simulation ist gezeitet und läuft in einer bestimmten Rate zur Realzeit. Diese Rate soll meist $1$ sein, bei der programmierte physikalische Vorgänge dieselbe Geschwindigkeit wie in der Realität annehmen. Die Simulation besitzt dadurch ihre eigene Simulationszeitbasis.
\item Verschiedene Arten von zu simulierenden Entitäten in einem Simulierten Raum (Gegner, Spielerfiguren, Projektile, Geräusche) mit unterschiedlichem Verhalten/unterschiedlicher Physik.
\item Interaktionen zwischen Entitäten (Kollisionen, Anti-Clipping)
\item Einflüsse durch Bediener/Spieler (Bewegung der Spielerfigur, Erzeugung von Projektilen)
\item Grafische Ausgabe in Echtzeit zu einer dreidimensionalen Perspektive
\end{itemize}

Die Simulation von Simulationsinhalten erfolgt in Schritten, in denen ein Inhalt von einem bestehenden Zustand auf einen Zustand zu fortgeschrittener Simulationszeit verändert wird. Ein solcher Zeitschritt wird als Tick bezeichnet (siehe Appendix \ref{sec:tick} oder vgl. Quelle \ref{tick}).\\
Die Anzeige der Simulation erfolgt durch Extrapolation des letzten bekannten Zustands der Simulationsinhalte. Die Anzeige führt dabei keine Änderungen auf den Simulationsinhalten durch. Wir bezeichnen den Prozess sowie sein Ergebnis als Frame (vgl. Appendix \ref{sec:tick}).
Die Framerate kann ungleich der Tickrate sein, um unabhängig von den Fähigkeiten des Anzeigesystems eine flüssiges Bild zu generieren. Da in einem Frame keine Simulationsinhalte geändert werden, sind die Simulationinhalte komplett unabhängig zur Anzeige.\\

In vielen kommerziellen Produkten in der Videospielbranche wird diese Trennung nicht sauber vollzogen.\\
Es folgen Inkonsistenzen in der Simulation in Abhängigkeit der Framerate.
(Beispiel in \glqq The Elder Scrolls V: Skyrim\grqq falsche physikalische Berechnungen auf Grund zu hoher Frameraten lernen Mammuts das Fliegen \cite{flying-fucking-mammoths})
Es existiert dort dann meist eine maximale Framerate.
Dieser Umstand scheint die Problematik mitzuführen, dass viele Videospielhersteller, vor allem im Konsolenbereich, Frameraten immer weiter nach unten limitieren, um ihre Produkte umzusetzen, was die Qualität erheblich senkt (Beispiele \cites{skyrim-physics-cap-and-fix, dark_souls-physics-cap-and-fix})
(entgegen der Meinung ihrer Marketingabteilungen, mit einer Menge negativer Presse, Beispiele \cites{morecinematic00, morecinematic01}
) und dazu führt dass Benutzer ihre u.U.~teure, fähige Hardware nicht ausnutzen können und sich mit schlechter Bildqualität zufrieden geben müssen.\\
Ein Umstand, der oft auf umständlichem Weg durch fähige Konsumenten selbstständig gelöst wird (vgl. \cites{skyrim-physics-cap-and-fix, dark_souls-physics-cap-and-fix})
Die Trennung von Physik und Grafik scheint eine grundsätzliche Designentscheidung zu sein, die in vielen kommerziell genutzten Engines fehlt.

Die beschriebene Anwendung funktioniert zu Beginn des Projektes lokal mit einem Benutzer. Es ist Ziel dieser Arbeit möglichst volle Funktionalität auf mehrere Benutzer, welche mit einer Netzwerkverbindung verbunden sind zu erweitern und dabei möglichst wenige negative Seiteneffekte zu erzeugen.

\subsection{Hergang}
Der Entwicklungsprozess in diesem Projekt umfasst das Erreichen bestimmter Meilensteine, die konstruktiv auf das Endziel hinführen sollen. Dieser Vorgang wird gewählt, um die Gesamtkomplexität des Endziels zu mitigieren. Durch die Abhängigkeit, die das Projekt vom Status quo des Vorprojektes hat, kann so dieses sukzessive an die hier gestellten Anforderungen angepasst werden.\\
Der Hergang lautet wie folgt:
\begin{enumerate}
\item Aufarbeitung des Status Quo\\
Umsetzung eines Designs im bestehenden Basisprojekt, das mit weiteren in diesem Projekt zu entwickelnden Features kompatibel ist.
\item Implementierung einer Fernbedienung\\
Fernbedienung als erstes netzwerkabhängiges Feature mit Einwegkommunikation, die keine Synchronisation benötigt.
\item Erneute Aufarbeitung
\item Implementierung der Echtzeitsynchronisation zwischen Simulationen\\
Auf zwei über Netzwerk verbundenen Maschinen wird jeweils eine Simulationsinstanz ausgeführt, deren Inhalte Synchronisiert werden.
\end{enumerate}



\section{Kontext \& Terminologie}

Implementierungstechnisch schränkt eine Rechenmaschine die mathematischen Zahlenräume ein:
\begin{itemize}
\item Die Reelen Zahlen $\mathbb{R}$ beschränken sich auf Floating-Point-Datentypen, welche hier im weiteren mit $\mathcal{F} \subset \mathbb{R}$ bezeichnet werden
\item Integern $\mathbb{Z}$ sind maschinell in ihrer Darstellungsgröße beschränkt. Diese beschränkte Menge an Integern wird $\mathcal{I} \subset \mathbb{Z}$ genannt
\end{itemize}


\subsection{Zeit}
\label{sec:time}
\def\finite#1{\ooalign{\hfil$\mapstochar\mkern 3mu\mapstochar\mkern 5mu$\hfil\cr$#1$}}

Jede Simulationsinstanz, d.h. eine Simulation auf einer konkreten Maschine, läuft gezeitet ab. Es existieren je zwei relevante zeitliche Sequenzen:
\begin{enumerate}
\item Die durch maschinelle Abtastung diskrete Realzeit der echten Welt\\
	\begin{itemize}
	\item $T_r:=\langle t_{epoch}, ... , t_{max}\rangle ; t_{epoch}, t_{max}$ als minimal, bzw.~maximal darstellbare Zeit
	\item die in einer maschinellen Genauigkeit in Mikrosekunden $$\epsilon_t:=10^{-6}s ; \forall c \in \mathbb{Z}: T_r(c) + \epsilon_t = T_r(c+1)$$ gegeben ist
	\item und immer monoton wächst $\forall c \in \mathbb{Z}: T_r(c) < T_r(c+1)$. 
	\end{itemize}
	

\item Die Simulationszeit $T_s:=\langle t_{start}, ... , t_{end}\rangle; t_{start}, t_{end}$ als Start- und Endzeitpunkt der Simulation.
	\begin{itemize}
	\item Zwischen den beiden Zeitbasen besteht eine totale, nicht-injektive, surjektive Abbildung $\mathcal{T}:T_r \twoheadrightarrow T_s$
	\item Die Simulationszeit ist dadurch relativ zur Realzeit definiert $T_s:=\langle\mathcal{T}\rangle$
	\item Um die Kontinuität der Zeit herzustellen wird weiter eine Zeittate $r_t:T_r\mapsto\mathcal{F}$ definiert, welche das relative verstreichen der Zeit in der Simulation steuert. 
	$$\forall t_{r0},  t_{r1} \in T_r ; t_{diff}=t_{r1}-t_{r0} :\mathcal{T}(t_{r1}) = \mathcal{T}(t_{r0}) + t_{diff}*r_t(t_{r1})$$ unter der Vorraussetzung, dass die Rate während aller Zeiten zwischen $t_{r0}$ und $t_{r1}$ gleich bleibt $\forall t \in [ t_{r0},t_{r1}]: r_t( t_{r0}) = r_t(t)$. Wird die Eigenschaft des Zeitflusses in der festgelegten Rate verletzt, werden die aktuellen Echtzeitanforderungen verletzt. \\
	Soll die Rate also geändert werden muss dies zu festgelegten Umschaltpunkten geschehen, welche die Simulationszeit in Zeitbereiche trennen, zwischen denen keine Berechnungsvorgänge zuverlässig durchführbar sind.\\
Beispiele für Raten sind 
\begin{itemize}
\item $r_t(x) = 1 \Leftrightarrow$ Simulation synchron zur Echtzeit
\item $r_t(x) = 0 \Leftrightarrow$ Simulation ist pausiert/läuft nicht
\end{itemize}
Technisch wird für große $|r_t|$ die Simulation schwierig, da viele Vorgänge schnell simuliert werden müssten. Diese werden daher vermieden.\\
Theoretisch kann die Rate auch negative Werte annehmen. Die Simulationszeit würde dann rückwärts laufen. Dieses Verhalten ist technisch durch die monoton steigende Realzeit nicht leicht in konsistenter Weise umzusetzen, da Realzeiteinflüsse durch Tasteneingaben existieren und soll daher hier ebenfalls vermieden werden.
	\end{itemize}
\end{enumerate}	


\subsection{Tick \& Frame}

Die Simulation behandelt das Verstreichen von Zeit in Zeitschritten, während dem der interne Zustand der Simulation, bzw.~der simulierten Objekte, zu einem zeitlich neuen Zustand aktualisiert wird.
Dieser Zeitschritt, bzw.~Verarbeitungsschritt, wird oft als Tick bezeichnet.\\

Es ist besonders anzumerken, das der Begriff des Ticks sich ausschließlich auf das Voranschreiten der Simulation bezieht und nicht dem Anzeigen einer Szene. Die Äquivalente Bezeichnung im Kontext der Grafik wird als Frame bezeichnet, in welchem eine Szene (ein Grafischer Zustand der Simulation) gerendert wird. Es besteht Verwechslungsgefahr. Von beiden Größen können Raten $r_{tick}, r_{frame}$ angegeben werden (üblicherweise in Tick/Frame pro Realzeitsekunde). Praktisch kann eine Grafikengine durch Inter- oder Extrapolation unterschiedliche Raten erreichen.\\
Wir definieren die Menge der Ticks $\delta:T_r^2; \delta:=\{\delta_1, \delta_2, ...\}$ anhand ihrer Start- und Endzeitpunkte in Echtzeit $\delta_i := (\delta_{i0}, \delta_{i1})$ und erweitert die zu einem Tick gehörenden Zeitpunkte als $\delta_{id}; d \in [0,1]$. Die Inklusivität/Exklusivität muss in bestimmten Berechnungskontexten manchmal angepasst werden um bei sukzessiven Ticks doppelte Behandlungen von Ereignissen zu vermeiden. Diese Einschätzung sei für jeden Kontext individuell zu vollziehen.\\
Es gilt außerdem die Kontinuität der Zeit auch bei Ticks $\delta_{j1} = \delta_{(j+1)0}$, d.h. ein Tick beginnt am Endzeitpunkt des Vorherigen.\\
Durch die Abbildung $\mathcal{T}$ erhält der Tick eine Entsprechung in Simulationszeit.\\
Ist im aktuellen Kontext nur ein Tick $\delta_i$ von belang wird auch die Terminologie $t_d =\mathcal{T}(\delta_{id})$, also $t_0$ für die Tickbeginn und $t_1$ für das Ende in Simulationszeit, verwendet.\\
Man kann weiter die Sequenz $\Upsilon_{\delta i} = \langle t_0, ...,  t_1\rangle$ als die zusammenhängende Partition der Simulationszeitsequenz $T_s$ denotieren, welche die geordneten Zeitpunkte eines Ticks in Simulationszeit enthält.\\
Die in Abschnitt~\ref{sec:time} beschriebenen möglichen Umschaltzeiten zur Änderung der Zeitflussrate in der Simulation werden auf die Grenzen von Ticks gelegt.\\
Die Größe der Zeitdifferenz $t_1 - t_0$ unterliegt meist Einschränkungen. Bestimmte Simulationsalgorithmen wie z.B. die Methode der kleinen Schritte erfordert für eine bestimmte Genauigkeit eine maximale Schrittgröße. Die verfügbare Rechenleistung hingegen beschränkt die Tickrate nach oben. Reicht die Berechnungszeit während eine Ticks nicht um den Status der Simulation von $t_0$ auf $t_1$ zu aktualisieren, läuft die Simulation langsamer als die reale Zeit. Die Echtzeitanforderung ist dann verletzt. Oft wird die Tickrate als Konstante festgelegt, in diesem Projekt ist jedoch nur eine Mindestrate festgelegt.


\subsection{Raum}
\label{sec:space}
Der geforderte 3D Raum kann durch 3-dimensionale Vektoren $\in \mathcal{F}^3$ in der Einheit Meter beschrieben werden.\\
Durch die Werteverteilung in $\mathcal{F}$ treten jedoch bei großen Räumen für Positionen mit großer Entfernung zum Ursprung $O$ Genauigkeitsdefizite auf, die zur Verletzung von Genauigkeitsanforderung führen können. Mögliche Floating-Point-Werte liegen dabei dichter beieinander, je näher am Ursprung \cite{floatdistribution}. Ein Beispiel für die Auswirkungen dieses Sachverhalts in Simulationen kann in der Quelle \cite{floatdistributionexample} betrachtet werden.\\
Physikalische Prozesse berechnet auf Basis von Positionen in $\mathcal{F}^3$ können daher inkonsistent in Abhängigkeit zum Ort im Raum sein.\\
Das Problem wird hier durch einen neuen Längendatentypen $\mathcal{S} : \mathcal{I} \times \mathcal{F}$ gelöst, welcher den Raum zunächst gleichmäßig durch $\mathcal{I}$ aufteilt und indiziert und $\mathcal{F}$ als Offset innerhalb seines Raumteils verwendet. Es wird daher eine Größe der initialen Aufteilung $size_{grid}$ definiert.\\
Die Umrechnung zu Metern ist dann: $$ meter: \mathcal{S} \mapsto \mathcal{F};  meter((i, f)) = i * size_{grid} + f * size_{grid}$$ 
Typischerweise gilt $f \in [0;1[$, um eine eindeutige Repräsentation für einen beschriebenen Punkt zu erhalten.

Diese Darstellung hat folgende weitere Vorteile
\begin{itemize}
\item Einfache Implementierung
\item Schnelle Indizierung der durch $\mathcal{I}$ indizierten Raumanteile für raumaufteilende Teile-und-Herrsche-Algorithmen
\end{itemize}

Absolute Positionen im Raum werden demnach mit Vektoren $s\in\mathcal{S}^3$ dargestellt. Für Berechnungen von Interaktionen zwischen Objekten werden Positionen zunächst relativiert, d.h. Positionen $p \in \mathcal{S}$ sollen zu $p_0$ relativ gesetzt werden, dann sind die relativen Positionen $p' = p - p_0$. Diese werden dann in in die für lokale Interaktionen sinnhafte  Einheit Meter $\mathcal{F}^3$ umgewandelt, um darauf Berechnungen durchzuführen. Man geht dabei davon aus, das relative Strecken zwischen Objekten kurz genug sind, sodass die Genauigkeitsänderung in $\mathcal{F}$ vernachlässigbar ist.\\
Effektiv ist dabei die Eigenschaft $\mathcal{F}\subset\mathcal{S}$ nicht gefordert, auch wenn sie in der in diesem Projekt verwendeten Implementierung prinzipiell gilt.

Implementierungstechnisch bestehen verschiedene Räume je nach Anwendungsfall, in denen Objekte durch Relativierung, Längenumrechnung und Transformation dargestellt werden.

\begin{enumerate}
\item Worldspace $= \mathcal{S}^3$ absolute Positionen von Objekten
\item Cameraspace $= \mathcal{F}^3$, Ursprung $O$ ist die Position der Kamera zum Rendern einer Szene, Objekte werden zur Kamera relativiert.
\item Viewspace $= \mathcal{F}^3$, Verzerrung durch die Kameralinse, um einen Blickwinkel auf einen Bildschirm anzupassen.
\item Objektspace,$= \mathcal{F}^3$, Ursprung ist der vom Modell definierten Mittelpunkt eines Objektes (Massenmittelpunkt), zur Verarbeitung von physikalischen Objektinteraktionen wird ein Objekt zu einem anderen Objekt relativiert.
\end{enumerate}

Auf diese Weise gelingt es selbst extreme absolute Entfernungen und Geschwindigkeiten im relativen akkurat zu behandeln.

\subsection{Entitäten}
\label{sec:entity}
Entitäten sind der atomare Inhalt der Simulation. Es handelt sich dabei um eine Abstraktion für \glqq Etwas\grqq  in der Simulation. In der Realität suchen Physiker immer weiter \glqq Das [sie] erkenne[n], was die Welt im Innersten zusammenhält, ..\grqq [- J.W.v. Goethe, Faust, \glqq Nacht\grqq, Vers. 832]. Was in der Realität eine offene Frage ist, obwohl seit den Zeiten von Goethe in der Quantenphysik durchaus Fortschritte gemacht worden sind, muss für die Simulation jedoch eindeutig beantwortet werden, um eine quantifizierbare Menge an Instanzen eines Konzepts zu haben, die dann Simuliert werden können.\\
Quantenmechanische Prozesse sind, um die Illusion einer realen Welt aus der Perspektive eines Menschen herzustellen, um einiges zu rechenaufwändig für unsere Anforderungen. Man setzt die Abstraktion der atomaren Simulationseinheit also höher an, was praktisch zu vielen Spezialisierungen führt. Die gemeinsame Basis dieser Spezialisierungen wird hier \textit{Entity}( dt. Entität ) genannt.\\

Verschiedenartige Beispiele:
\begin{enumerate}
\item klassisch: 3D-Objekt mit einer Position, Rotation und Form, welches sich in Abhängigkeit der Zeit mit einer Geschwindigkeit bewegt.\\
\item formlose: Entität kann z.B.~ein Geräusch sein, welches in der Simulation durch seinen Quellort abstrahiert ist, sich ebenfalls bewegen kann, aber keine Rotation besitzen muss, oder zugehörig zu einer anderen Entität ist.
\item unphysikalisch: Logische Entitäten, die beispielsweise die Anwesenheit eines Objektes in einem bestimmten Raumabschnitt prüft und einer Subroutine das Ergebnis übermittelt.
\end{enumerate}
Die Abstraktion der Entität dient demnach als Schnittstelle für die Simulation, um Zeit (während eines Ticks) auf atomare Bestandteile der Simulation anzuwenden, und als Implementierungsplattform für die verschiedenen Anwendungsanforderungen der Simulation, bzw.~dem Verhalten des Simulationsinhalts.

\subsection{Interaktion}
Interaktionen werden zwischen Entitäten durchgeführt. Der Auslöser für eine Interaktion ist dabei an bestimmte Bedingungen geknüpft. Die wiederholte Auflösung der Bedingungen, sowie die Berechnung der konkreten Interaktion stellt einen wesentlichen Bestandteil der Aufgabe der Simulation dar.\\
Ein klassisches Beispiel einer Interaktion ist die physikalische Kollision zweier Objekte. Dabei muss u.U.~ der Hergang der Kollision, so wie eine passende Reaktion der Kollisionspartner ermittelt werden.\\
Als initiale Bedingung für das Auftreten einer Interaktion werden in dieser Simulation AABBs über dem Einflussgebiet einer Entität kollidiert, um entsprechend Interaktionspaare zu finden.





\label{sec:terminology}
Implementierungstechnisch schränkt eine Rechenmaschine die mathematischen Zahlenräume ein:
\begin{itemize}
\item Die Reelen Zahlen $\mathbb{R}$ beschränken sich auf Floating-Point-Datentypen, welche hier im weiteren mit $\mathcal{F} \subset \mathbb{R}$ bezeichnet werden
\item Integern $\mathbb{Z}$ sind maschinell in ihrer Darstellungsgröße beschränkt. Diese beschränkte Menge an Integern wird $\mathcal{I} \subset \mathbb{Z}$ genannt
\end{itemize}


\subsection{Zeit}
\label{sec:time}
\def\finite#1{\ooalign{\hfil$\mapstochar\mkern 3mu\mapstochar\mkern 5mu$\hfil\cr$#1$}}

Jede Simulationsinstanz, d.h. eine Simulation auf einer konkreten Maschine, läuft gezeitet ab. Es existieren je zwei relevante zeitliche Sequenzen:
\begin{enumerate}
\item Die durch maschinelle Abtastung diskrete Realzeit der echten Welt\\
	\begin{itemize}
	\item $T_r:=\langle t_{epoch}, ... , t_{max}\rangle ; t_{epoch}, t_{max}$ als minimal, bzw.~maximal darstellbare Zeit
	\item die in einer maschinellen Genauigkeit in Mikrosekunden $$\epsilon_t:=10^{-6}s ; \forall c \in \mathbb{Z}: T_r(c) + \epsilon_t = T_r(c+1)$$ gegeben ist
	\item und immer monoton wächst $\forall c \in \mathbb{Z}: T_r(c) < T_r(c+1)$. 
	\end{itemize}
	

\item Die Simulationszeit $T_s:=\langle t_{start}, ... , t_{end}\rangle; t_{start}, t_{end}$ als Start- und Endzeitpunkt der Simulation.
	\begin{itemize}
	\item Zwischen den beiden Zeitbasen besteht eine totale, nicht-injektive, surjektive Abbildung $\mathcal{T}:T_r \twoheadrightarrow T_s$
	\item Die Simulationszeit ist dadurch relativ zur Realzeit definiert $T_s:=\langle\mathcal{T}\rangle$
	\item Um die Kontinuität der Zeit herzustellen wird weiter eine Zeittate $r_t:T_r\mapsto\mathcal{F}$ definiert, welche das relative verstreichen der Zeit in der Simulation steuert. 
	$$\forall t_{r0},  t_{r1} \in T_r ; t_{diff}=t_{r1}-t_{r0} :\mathcal{T}(t_{r1}) = \mathcal{T}(t_{r0}) + t_{diff}*r_t(t_{r1})$$ unter der Vorraussetzung, dass die Rate während aller Zeiten zwischen $t_{r0}$ und $t_{r1}$ gleich bleibt $\forall t \in [ t_{r0},t_{r1}]: r_t( t_{r0}) = r_t(t)$. Wird die Eigenschaft des Zeitflusses in der festgelegten Rate verletzt, werden die aktuellen Echtzeitanforderungen verletzt. \\
	Soll die Rate also geändert werden muss dies zu festgelegten Umschaltpunkten geschehen, welche die Simulationszeit in Zeitbereiche trennen, zwischen denen keine Berechnungsvorgänge zuverlässig durchführbar sind.\\
Beispiele für Raten sind 
\begin{itemize}
\item $r_t(x) = 1 \Leftrightarrow$ Simulation synchron zur Echtzeit
\item $r_t(x) = 0 \Leftrightarrow$ Simulation ist pausiert/läuft nicht
\end{itemize}
Technisch wird für große $|r_t|$ die Simulation schwierig, da viele Vorgänge schnell simuliert werden müssten. Diese werden daher vermieden.\\
Theoretisch kann die Rate auch negative Werte annehmen. Die Simulationszeit würde dann rückwärts laufen. Dieses Verhalten ist technisch durch die monoton steigende Realzeit nicht leicht in konsistenter Weise umzusetzen, da Realzeiteinflüsse durch Tasteneingaben existieren und soll daher hier ebenfalls vermieden werden.
	\end{itemize}
\end{enumerate}	


\subsection{Tick \& Frame}

Die Simulation behandelt das Verstreichen von Zeit in Zeitschritten, während dem der interne Zustand der Simulation, bzw.~der simulierten Objekte, zu einem zeitlich neuen Zustand aktualisiert wird.
Dieser Zeitschritt, bzw.~Verarbeitungsschritt, wird oft als Tick bezeichnet.\\

Es ist besonders anzumerken, das der Begriff des Ticks sich ausschließlich auf das Voranschreiten der Simulation bezieht und nicht dem Anzeigen einer Szene. Die Äquivalente Bezeichnung im Kontext der Grafik wird als Frame bezeichnet, in welchem eine Szene (ein Grafischer Zustand der Simulation) gerendert wird. Es besteht Verwechslungsgefahr. Von beiden Größen können Raten $r_{tick}, r_{frame}$ angegeben werden (üblicherweise in Tick/Frame pro Realzeitsekunde). Praktisch kann eine Grafikengine durch Inter- oder Extrapolation unterschiedliche Raten erreichen.\\
Wir definieren die Menge der Ticks $\delta:T_r^2; \delta:=\{\delta_1, \delta_2, ...\}$ anhand ihrer Start- und Endzeitpunkte in Echtzeit $\delta_i := (\delta_{i0}, \delta_{i1})$ und erweitert die zu einem Tick gehörenden Zeitpunkte als $\delta_{id}; d \in [0,1]$. Die Inklusivität/Exklusivität muss in bestimmten Berechnungskontexten manchmal angepasst werden um bei sukzessiven Ticks doppelte Behandlungen von Ereignissen zu vermeiden. Diese Einschätzung sei für jeden Kontext individuell zu vollziehen.\\
Es gilt außerdem die Kontinuität der Zeit auch bei Ticks $\delta_{j1} = \delta_{(j+1)0}$, d.h. ein Tick beginnt am Endzeitpunkt des Vorherigen.\\
Durch die Abbildung $\mathcal{T}$ erhält der Tick eine Entsprechung in Simulationszeit.\\
Ist im aktuellen Kontext nur ein Tick $\delta_i$ von belang wird auch die Terminologie $t_d =\mathcal{T}(\delta_{id})$, also $t_0$ für die Tickbeginn und $t_1$ für das Ende in Simulationszeit, verwendet.\\
Man kann weiter die Sequenz $\Upsilon_{\delta i} = \langle t_0, ...,  t_1\rangle$ als die zusammenhängende Partition der Simulationszeitsequenz $T_s$ denotieren, welche die geordneten Zeitpunkte eines Ticks in Simulationszeit enthält.\\
Die in Abschnitt~\ref{sec:time} beschriebenen möglichen Umschaltzeiten zur Änderung der Zeitflussrate in der Simulation werden auf die Grenzen von Ticks gelegt.\\
Die Größe der Zeitdifferenz $t_1 - t_0$ unterliegt meist Einschränkungen. Bestimmte Simulationsalgorithmen wie z.B. die Methode der kleinen Schritte erfordert für eine bestimmte Genauigkeit eine maximale Schrittgröße. Die verfügbare Rechenleistung hingegen beschränkt die Tickrate nach oben. Reicht die Berechnungszeit während eine Ticks nicht um den Status der Simulation von $t_0$ auf $t_1$ zu aktualisieren, läuft die Simulation langsamer als die reale Zeit. Die Echtzeitanforderung ist dann verletzt. Oft wird die Tickrate als Konstante festgelegt, in diesem Projekt ist jedoch nur eine Mindestrate festgelegt.


\subsection{Raum}
\label{sec:space}
Der geforderte 3D Raum kann durch 3-dimensionale Vektoren $\in \mathcal{F}^3$ in der Einheit Meter beschrieben werden.\\
Durch die Werteverteilung in $\mathcal{F}$ treten jedoch bei großen Räumen für Positionen mit großer Entfernung zum Ursprung $O$ Genauigkeitsdefizite auf, die zur Verletzung von Genauigkeitsanforderung führen können. Mögliche Floating-Point-Werte liegen dabei dichter beieinander, je näher am Ursprung \cite{floatdistribution}. Ein Beispiel für die Auswirkungen dieses Sachverhalts in Simulationen kann in der Quelle \cite{floatdistributionexample} betrachtet werden.\\
Physikalische Prozesse berechnet auf Basis von Positionen in $\mathcal{F}^3$ können daher inkonsistent in Abhängigkeit zum Ort im Raum sein.\\
Das Problem wird hier durch einen neuen Längendatentypen $\mathcal{S} : \mathcal{I} \times \mathcal{F}$ gelöst, welcher den Raum zunächst gleichmäßig durch $\mathcal{I}$ aufteilt und indiziert und $\mathcal{F}$ als Offset innerhalb seines Raumteils verwendet. Es wird daher eine Größe der initialen Aufteilung $size_{grid}$ definiert.\\
Die Umrechnung zu Metern ist dann: $$ meter: \mathcal{S} \mapsto \mathcal{F};  meter((i, f)) = i * size_{grid} + f * size_{grid}$$ 
Typischerweise gilt $f \in [0;1[$, um eine eindeutige Repräsentation für einen beschriebenen Punkt zu erhalten.

Diese Darstellung hat folgende weitere Vorteile
\begin{itemize}
\item Einfache Implementierung
\item Schnelle Indizierung der durch $\mathcal{I}$ indizierten Raumanteile für raumaufteilende Teile-und-Herrsche-Algorithmen
\end{itemize}

Absolute Positionen im Raum werden demnach mit Vektoren $s\in\mathcal{S}^3$ dargestellt. Für Berechnungen von Interaktionen zwischen Objekten werden Positionen zunächst relativiert, d.h. Positionen $p \in \mathcal{S}$ sollen zu $p_0$ relativ gesetzt werden, dann sind die relativen Positionen $p' = p - p_0$. Diese werden dann in in die für lokale Interaktionen sinnhafte  Einheit Meter $\mathcal{F}^3$ umgewandelt, um darauf Berechnungen durchzuführen. Man geht dabei davon aus, das relative Strecken zwischen Objekten kurz genug sind, sodass die Genauigkeitsänderung in $\mathcal{F}$ vernachlässigbar ist.\\
Effektiv ist dabei die Eigenschaft $\mathcal{F}\subset\mathcal{S}$ nicht gefordert, auch wenn sie in der in diesem Projekt verwendeten Implementierung prinzipiell gilt.

Implementierungstechnisch bestehen verschiedene Räume je nach Anwendungsfall, in denen Objekte durch Relativierung, Längenumrechnung und Transformation dargestellt werden.

\begin{enumerate}
\item Worldspace $= \mathcal{S}^3$ absolute Positionen von Objekten
\item Cameraspace $= \mathcal{F}^3$, Ursprung $O$ ist die Position der Kamera zum Rendern einer Szene, Objekte werden zur Kamera relativiert.
\item Viewspace $= \mathcal{F}^3$, Verzerrung durch die Kameralinse, um einen Blickwinkel auf einen Bildschirm anzupassen.
\item Objektspace,$= \mathcal{F}^3$, Ursprung ist der vom Modell definierten Mittelpunkt eines Objektes (Massenmittelpunkt), zur Verarbeitung von physikalischen Objektinteraktionen wird ein Objekt zu einem anderen Objekt relativiert.
\end{enumerate}

Auf diese Weise gelingt es selbst extreme absolute Entfernungen und Geschwindigkeiten im relativen akkurat zu behandeln.

\subsection{Entitäten}
\label{sec:entity}
Entitäten sind der atomare Inhalt der Simulation. Es handelt sich dabei um eine Abstraktion für \glqq Etwas\grqq  in der Simulation. In der Realität suchen Physiker immer weiter \glqq Das [sie] erkenne[n], was die Welt im Innersten zusammenhält, ..\grqq [- J.W.v. Goethe, Faust, \glqq Nacht\grqq, Vers. 382]. Was in der Realität eine offene Frage ist, obwohl seit den Zeiten von Goethe in der Quantenphysik durchaus Fortschritte gemacht worden sind, muss für die Simulation jedoch eindeutig beantwortet werden, um eine quantifizierbare Menge an Instanzen eines Konzepts zu haben, die dann Simuliert werden können.\\
Quantenmechanische Prozesse sind, um die Illusion einer realen Welt aus der Perspektive eines Menschen herzustellen, um einiges zu rechenaufwändig für unsere Anforderungen. Man setzt die Abstraktion der atomaren Simulationseinheit also höher an, was praktisch zu vielen Spezialisierungen führt. Die gemeinsame Basis dieser Spezialisierungen wird hier \textit{Entity}( dt. Entität ) genannt.\\

Verschiedenartige Beispiele:
\begin{enumerate}
\item klassisch: 3D-Objekt mit einer Position, Rotation und Form, welches sich in Abhängigkeit der Zeit mit einer Geschwindigkeit bewegt.\\
\item formlose: Entität kann z.B.~ein Geräusch sein, welches in der Simulation durch seinen Quellort abstrahiert ist, sich ebenfalls bewegen kann, aber keine Rotation besitzen muss, oder zugehörig zu einer anderen Entität ist.
\item unphysikalisch: Logische Entitäten, die beispielsweise die Anwesenheit eines Objektes in einem bestimmten Raumabschnitt prüft und einer Subroutine das Ergebnis übermittelt.
\end{enumerate}
Die Abstraktion der Entität dient demnach als Schnittstelle für die Simulation, um Zeit (während eines Ticks) auf atomare Bestandteile der Simulation anzuwenden, und als Implementierungsplattform für die verschiedenen Anwendungsanforderungen der Simulation, bzw.~dem Verhalten des Simulationsinhalts.

\subsection{Bounding-Volume}
Ein Bounding-Volume zu einem Objekt $o$ ist eine kompakte Menge $B_o \supseteq K_{o}$. $B_o$ kann als Hitbox fungieren.\\
Eine Bounding-Box ist ein spezielles Bounding-Volume in Form eines Quaders.\\
Eine in diesem Projekt extensiv verwendete, tiefere Spezialform der Bounding-Box ist die Axis-Alligned-Bounding-Box (AABB). Alle Kanten dieser Bounding-Box sind achsenparallel zu den Koordinatenachsen $\{(1,0,0), (0,1,0), (0,0,1)\}$ des 3D-Koordinatensystems.\\
Hier relevante Eigenschaften dieser sind:
\begin{itemize}
\item kleine Datenrepresentation:
		$$AABB_o \in \mathcal{S}^{3^2}; AABB_o = (BB_{min}, BB_{max}) = ((x_{min}, y_{min}, z_{min}), (x_{max}, y_{max}, z_{max}))$$.
		 In ihnen werden Minimal- und Maximalpositionen der AABB festgehalten.
		 Diese Positionen werden im Worldspace $\mathcal{S}^{3^2}$ angegeben, da AABBs hier die primäre vereinfachende Abstraktion sein sollen, die in der Simulation für Objekte verwendet wird. Die Angabe im Worldspace macht AABBs für Berechnungen im absoluten, zum Beispiel räumliche Partitionierung für Teile-und-Herrsche Algorithmen, direkt zugänglich.
		 Die theoretische, kompakte Punktemenge 
		 \begin{align*}
		 \mathcal{AABB}: \mathcal{S}^{3^2} \mapsto \mathcal{F}^3;
		 \mathcal{AABB} ((x_{min}, y_{min}, z_{min}), (x_{max}, y_{max}, z_{max})) = \\
		 \{meter((x_{min} + x * (x_{max} - x_{min}), y_{min} + y * (y_{max} - y_{min}),\\
		  z_{min} + z * (z_{max} - z_{min}))| x, y, z \in [0,1] \} 
		 \end{align*}
		 ist dann ein Bounding-Volume $\mathcal{AABB}(AABB_o) \supseteq K_o$
	\item Ermittlung einer minimalen AABB für ein Objekt durch Suche der Minima und Maxima der Ausdehnung eines Objektes in jeder Koordinatenachse:
	 $x_{min} = x : (x, y, z) \in V_o , \forall (x', y', z') \in V_o: x \leq x'; y \& z, min \& max $ analog.
	 Die Findung dieser Werte ist in $ \mathcal{O}(|V_o|) $.
		Bei rotierenden Objekten ist auch eine minimale AABB zu diesem Objekt einer Bewegung ausgesetzt, standardmäßig durch Neuermittlung der AABB($\mathcal{O}(n)$ pro Tick). Optimierungen für verschiedene Arten von Bewegung möglich (Positionsänderung, Skalierung, etc.), aber manchmal schwierig bis unmöglich (z.B. bei Rotation).\\
		Aus diesem Grund macht es auch Sinn, nicht-minimale AABBs zu wählen.
	\item Schnelle Kollisionsüberprüfung zwischen AABBs durch Vergleiche der Extrema $\mathcal{O}(1)$
\end{itemize}

AABBs, bzw. Bounding-Volumes generell, werden nicht ausschließlich für statische Objekte $K_{o,t}$ zu einem Zeitpunkt erstellt und verwendet. Es macht in bestimmten Kontexten zum Beispiel auch Sinn das komplette durchlaufene Volumen eines Objektes $\{K_{o,t'} | t' \in \Upsilon_{\delta_i}\}$ während eines Ticks durch ein Bounding-Volume zu abstrahieren.

Im Falle des Spiels Minecraft werden AABBs als finale Hitboxen verwendet (vgl. Abbildungen \ref{fig:mwhitbox}, \ref{fig:mphitbox}), welche jedoch scheinbar dem Kriterium $K_o \subseteq B_o$ zuwiderlaufen. Es muss an dieser Stelle zwischen der mathematischen Korrektheit einer Bounding-Box gegenüber einem gegebenen physikalischen Modell und der Designentscheidung gemacht werden, dass das sichtbare Modell nicht oder nur marginal die Grundlage des physikalischen Modells ist. In Minecraft ist die AABB die finale Hitbox  $H_{o, [min]} = K_o$ und definiert dadurch selbst das Modell $K_o$. Das Bounding-Box-Kriterium ist damit theoretisch erfüllt. Die Designentscheidung selbst soll an dieser Stelle nicht eingeschätzt werden.


\subsection{Interaktion}
Interaktionen werden zwischen Entitäten durchgeführt. Der Auslöser für eine Interaktion ist dabei an bestimmte Bedingungen geknüpft. Die wiederholte Auflösung der Bedingungen, sowie die Berechnung der konkreten Interaktion stellt einen wesentlichen Bestandteil der Aufgabe der Simulation dar.\\
Ein klassisches Beispiel einer Interaktion ist die physikalische Kollision zweier Objekte. Dabei muss u.U.~ der Hergang der Kollision, so wie eine passende Reaktion der Kollisionspartner ermittelt werden.\\
Als initiale Bedingung für das Auftreten einer Interaktion werden in dieser Simulation AABBs über dem Einflussgebiet einer Entität kollidiert, um entsprechend Interaktionspaare zu finden.

\subsection{Aktionen}
Aktionen werden von Entitäten durchgeführt und stellen eine unvorhersehbare Änderung einer Entität dar. Aktionen sind pro Entitätstyp definiert. Der Auslöser für eine Aktion ist dabei außerhalb der Definition der Entität zu suchen. Derzeit passt nur der menschliche Bediener auf diese Beschreibung, allerdings sind theoretisch auch andere Einflüsse, zum Beispiel durch eine Form von künstlicher Intelligenz denkbar, die nicht direkt als Teil der Entität modelliert wurde.


\subsection{Projektanteile/Neuleistungen}
Die im Abschnitt \ref{sec:terminology} behandelten Konzepte sind im Status-Quo zu Beginn des Projektes umgesetzt. Ihre Behandlung in diesem Bericht dient im Folgenden als Kontext.







\section{Hergang}
Der Entwicklungsprozess in diesem Projekt umfasst das Erreichen bestimmter Meilensteine, die konstruktiv auf das Endziel hinführen sollen. Dieser Vorgang wird gewählt, um die Gesamtkomplexität des Endziels zu mitigieren. Durch die Abhängigkeit, die das Projekt vom Status quo des Vorprojektes hat, kann so dieses sukzessive an die hier gestellten Anforderungen angepasst werden.\\
Der Hergang lautet wie folgt:
\begin{enumerate}
\item Aufarbeitung des Status Quo\\
Umsetzung eines Designs im bestehenden Basisprojekt, das mit weiteren in diesem Projekt zu entwickelnden Features kompatibel ist.
\item Implementierung einer Fernbedienung\\
Fernbedienung als erstes netzwerkabhängiges Feature mit Einwegkommunikation, die keine Synchronisation benötigt.
\item Erneute Aufarbeitung
\item Implementierung der Echtzeitsynchronisation zwischen Simulationen\\
Auf zwei über Netzwerk verbundenen Maschinen wird jeweils eine Simulationsinstanz ausgeführt, deren Inhalte Synchronisiert werden.
\end{enumerate}



\section{Aufarbeitung des Status Quo/Vorbereitung zur Fernsteuerung}
\section{Umsetzung der Fernsteuerung}

Die Fernsteuerung soll ausschließlich die Steuerung einer Spielfigur auf einem Remote ermöglichen. Ausgeschlossen sind dadurch die grafische Ausgabe an der Steuerungsapplikation und generell eine Rücksendung von Information vom Remote zur Steuerung.\\

\subsection{Formatierung von Benutzereingaben}
Zum Zweck der Fernsteuerung werden Benutzereingaben in ein übertragbares Format gebracht.\\
Es wird sich nach Abschnitt \ref{sec:transmission_formats} hier auf Grund der Echtzeitrelevanz auf die Übertragung von Statusinformation via UDP geeinigt.\\
Es wird zunächst in Betracht gezogen direkt Eingabeereignisse zu versenden. Eingabeereignisse sind
kleine Einheiten von Information, welche den Zustand oder die Änderung von Eingabegeräten angeben. Eingabeereignisse werden direkt von Bibliotheken ausgegeben, welche entsprechend Eingabegeräte dem Programm auf diese Weise zugänglich machen. Hier wird zu diesem Zweck SFML\cite{sfml} verwendet.\\
Allerdings entstehen hier einige Probleme:
\begin{itemize}
\item Eingabeereignisse bilden teilweise Stati und teilweise Änderungsinformation ab. Beispielsweise führt die Zustandsänderung einer Taste zu einem konsumierbaren Ereignis, was eine Änderungsinformation darstellt, welches selbst allerdings einen Status der Taste beinhält. Als ein anderes Beispiel existieren Ereignisse, welche die Differenz des Stands des Mausrades angeben, da dort keine absolute Skala verwendet wird. Die Probleme, welche aus diesen unterschiedlichen Formaten oder stattdessen der Verwendung von TCP entstehen wurden bereits in Abschnitt \ref{sec:transmission_formats} genauer beschrieben. Die Verwendung von rohen Eingabeereignissen erforderte also eine Umwandlung zu Statusinformation.
\item Eingabeereignisse sind teilweise clientseitig kontextbezogen. Vor allem die Verwendung von Ereignissen, welche von der Maus erzeugt werden, müssen oft unter Einbezug der Mausposition realisiert werden, welche relativ zur verwendeten Auflösung sowohl des verwendeten Monitors, als auch der Größe des Applikationsfensters abhängen. Beides sind Größen, auf welche die Applikation keinen authoritären Eingiff hat oder haben soll. Dazu kommen ebenfalls Probleme mit der Umsetzung verschiedener Maus-Modi, bei denen die Maus wahlweise als Zeigegerät oder als analoge Eingabemethode verwendet werden soll, was durch lokale Uminterpretierung der Mausereignisse realisiert wird.
\item Viele Eingabeereignisse stellen keine Information dar, welche den Remote, bzw. den Server tangieren. Größenveränderung des Client-Fensters, die Verwendung einer unbenutzten Taste, diese Dinge können als Information-Leakage interpretiert werden. Effektiv ist die Applikation unter diesen Umständen ein Keylogger.
\item Eingabeereignisse werden vielfach erzeugt und ändern oft nur marginal den Stand der Dinge. Es wird als verschwenderisch Angesehen eine Vielzahl gleichartiger Ereignisse unter Belastung der Verfügbaren Bandbreite zu versenden, wenn die meisten enthaltenen Ereignisse nur zu einer Überschreibung am Remote führen und effektiv keinen Effekt haben.
\item Um die verfügbare Bandbreite besser auszunutzen müssten Ereignisse gesammelt und in Gruppen versandt werden. Dadurch geht die Information über zeitliche Abstände zwischen Ereignissen weitestgehend verloren. Das Hinzufügen von Zeitstempeln zu Ereignissen wird als unnötiger Overhead sowohl für die Applikationen als auch die Übertragung angesehen.
\end{itemize}

Eingabeereignisse werden daher clientseitig in einen Kontrollstatus konsolidiert, eine Menge an Statusinformation, welche als Eingabeformat für die Remote-Simulation gilt, und zyklisch versendet werden kann. Die Größe dieser Menge ist vergleichsweise klein (derzeitig 64 Bytes).\\
Der Status umfasst ebenfalls nur Stati, welche für die Simulation von Interesse sind und bildet so die Bedienerintention gegenüber der Simulation, bzw. seiner Spielerfigur ab.\\
Einige Eingaben die ein Bediener tätigen kann, beziehen sich nicht auf Simulationsinhalte per se, sonders beispielsweise auf die Anpassung der Perspektive (erste und dritte Person) oder der Anzeige bestimmter Metriken der Simulation, z.B.~ Frames per Second, etc.. Diese können unabhängig vom Kontrollstatus clientseitig durch die Ereignisverarbeitung direkt verarbeitet werden.\\

Bei der Fernsteuerung ist zunächst die Rücksendung von Information vom Remote nicht erwünscht. Durch den Kontrollstatus als konsolidiertes Eingabeformat kann die unidirektionalität der Eingabe leichter versichert werden. Die Eingabe kann auf diese Weise zunächst auch erst einmal lokal getestet und verwendet werden, bevor die tatsächliche Fernsteuerung über Netzwerk umgesetzt wird.
Die Umsetzung einiger Features als unidirektionaler Status gestaltet sich jedoch nicht als trivial:
Zunächst waren zuvor einige Features ohne diese Anforderungen realisiert und erforderten einen Umbau, um einen unidirektionalen Status interpretieren zu können.
Außerdem sind einige Features strukturell unter den Anforderungen schwierig umzusetzen, bzw.~leiden sogar unter den Anforderungen:
\begin{itemize}
\item Die Betätigung eines Auslösers, z.B.~zum Schießen in der Shooter-Applikation ist starken Echtzeitanforderungen unterlegen. Bei längeren Latenzzeiten kann z.B. das klicken der Maus mehrfach geschehen. Wie oft und wie lange wurde der Schalter nun betätigt? Bei den möglicherweise anliegenden Latenzen ist eine dynamische Abtastrate durch einen nicht-dynamischen Status nicht abbildbar. Weiter ist die Abtastrate des Eingabegeräts ebenfalls beschränkt, wir schätzen diese allerdings als so hoch ein, dass dies irrelevant für unsere Überlegungen ist. An dieser Stelle sind unter den gegebenen Umständen die Fähigkeiten der Applikation beschränkt.
\item Die Selektion aus einer Menge von selektierbaren Dingen.\\
Ist die Menge der selektierbare Dinge bekannt und statisch, kann die Selektion durch einen einfachen Integer-Status  in einem Restraum dargestellt werden. Ist die Menge auszuwählender Elemente jedoch dynamisch, wird ohne die Möglichkeit der Rückkommunikation dieser Anzahl die Umsetzung dieses Features weitaus schwieriger, bzw. Anforderungen an den Selektionsmechanismus müssen grundlegend überdacht werden.
%% TODO Lukas erklär dein bums
\end{itemize}

Einige Konzepte lassen sich jedoch vergleichbar leicht umsetzen.
\begin{itemize}
\item Absolute Position der Benutzerfigur
\item Absolute Ausrichtung der Benutzerfigur
\item Bewegungsintention des Bedieners
\item etc.
\end{itemize}

\subsection{Umsetzung im Design}

\begin{figure}
\centering
\begin{subfigure}[b]{0.3\textwidth}
\centering
\resizebox{\linewidth}{!}{
\begin{tikzpicture}[framed, thick,scale=1, every node/.style={scale=1}]

\begin{class}{Simulation}{0,-2}
\end{class}
\begin{class}{User}{0,0}
\end{class}
\begin{class}{LocalUser}{0,2}
\inherit{User}
\end{class}

\unidirectionalAssociation{User}{1}{}{Simulation}{1}{}

\end{tikzpicture}
}
\caption{UML-Diagramm zur Umsetzung eines Lokalen Anwenders der Simulation}
\label{fig:local}

\end{subfigure}
\begin{subfigure}[b]{0.3\textwidth}
\centering
\resizebox{\linewidth}{!}{
\begin{tikzpicture}[framed, thick,scale=1, every node/.style={scale=1}]

\begin{class}{Simulation}{0,-2}
\end{class}
\begin{class}{User}{0,0}
\end{class}
\begin{class}{RemoteControledUser}{0,2}
\inherit{User}
\end{class}
\begin{class}{RemoteControlSender}{0,4}
\end{class}

\unidirectionalAssociation{User}{1}{}{Simulation}{1}{}
\unidirectionalAssociation{RemoteControlSender}{1}{}{RemoteControledUser}{1}{}

\end{tikzpicture}
}
\caption{UML-Diagramm zur Umsetzung des fernsteuernden Anwenders}
\label{fig:remotecontrol_indiv}
\end{subfigure}
\label{fig:remotecontrol_design}
\end{figure}

Unter dem neugewonnenen Design können verschiedene Arten von Benutzern definiert werden. Es wird der zunächst der lokale Benutzer definiert um den alten Verwendungszweck abzubilden, und dann eine neue Art Benutzer hinzugefügt, welcher einen Benutzer über eine Netzwerkverbindung abstrahiert. Man vergleiche hierzu die Abbildungen \ref{fig:remotecontrol_design}.\\
Die Abstraktion eines Benutzers kann in beiden Fällen das Eingabeformat des Kontrollstatus als gemeinsame Schnittstelle bereitstellen. Der Kontrollstatus wird je nach Art des Benutzers entweder lokal durch Ereignisse, oder über eine Netzwerkverbindung aktualisiert. 
In der Simulation wird pro Benutzer egal welcher Art eine Benutzerentität angelegt.
Diese werden dem Benutzer zugewiesen und sind fähig dessen Kontrollstatus entsprechend zu interpretieren.


\section{Problemdefinition}
Der Begriff Kollisionserkennung (\textit{eng. collision detection}) scheint, je nach Anwendungsfall, ein spezifisches Problem zu sein und beschreibt eine Sammlung verschiedener Teilprobleme.\\
Wir beschränken uns für die Zwecke dieses Projektes auf Folgende:
\begin{enumerate}
\item Die Ermittlung der Menge von Objektpaaren $C \subseteq OBJ^2$ aus einer Menge von sich bewegenden Objekten $OBJ$ in einem Raum, welche sich innerhalb eines Zeitraums, hier einem Tick $\delta_i$, überschneiden.
$$C = \{(o_0, o_1) | o_0, o_1 \in OBJ; t\in \Upsilon_{\delta_i}; G_{o_0, t} \cap G_{o_1, t} \neq \emptyset\}$$
\item Die Ermittlung von Informationen über den Hergang einer Kollision.
Die genauen Anforderungen, welche Informationen ermittelt werden müssen, gehen aus den Verwendungszwecken hervor (vgl.~\ref{sec:usages}). 
\end{enumerate}

Während mögliche Anforderungen von Verwendungszwecken zwar betrachtet werden müssen, sollen die Verwendungszwecke selbst in diesem Projekt konkret nicht realisiert werden.

Wir schätzen an dieser Stelle modellgenaue Kollisionen als schwierig ein. In Teilproblem 1 entsteht daraus das Problem einer schnell steigenden Komplexität, da die Anzahl der möglichen Kollisionspaare quadratisch wächst $|C|\leq |OBJ|^2$.\\
Das Teilproblem 1 wird daher in zwei separaten Schritten behandelt.
\begin{itemize}
\item[1.1] Elegante und effiziente Vorfilterung von Objekten $OBJ$, um die modellgenaue Überprüfung von großen Mengen von Objektpaaren in $C$ zu vermeiden.
\item[1.2] Die abstrakte Behandlung des modellgenauen Kollisionsproblems bei genau zwei Objekten. Dabei werden Algorithmen gewählt, welche die für Teilproblem 2 benötigten Informationen mitliefern.
\end{itemize}

Für beide dieser Schritte sollen im Folgenden Algorithmen beschrieben werden.\\
\\
\label{sec:physical_realism}
Es ist weiter üblicherweise ein Ziel von physikalischen Simulationen einen realen Sachverhalt möglichst akkurat und konform mit dem etablierten physikalischen Verständnis umzusetzen. Diese Anforderungen müssen für den bestehenden Verwendungszweck der Simulation relativiert werden.\\
In Videospielen dient die akkurate Umsetzung nur der Immersion des Konsumenten,~d.h. der Konsument soll der Illusion von absolutem Realismus ausgesetzt sein, während absoluter Realismus aber schwer bis gar nicht umsetzbar sein kann. Je nach Umstand ist sogar der Bruch der Illusion in Maßen nicht kritisch. Es sollte dem Konsumenten möglich sein mit Hilfe seines vorhandenen physikalischen Verständnisses ein Verständnis für die Vorgänge der Simulation intuitiv entwickeln zu können.\\
Videospiele sind im Kontext der Simulationsgenauigkeit generell sehr vergebend. Zum Vergleich: Simulationen mit zu niedrigen Fehlertoleranzen in z.B.~der Industrierobotik können zu erheblichem Sach- oder gar Personenschaden führen.
Dem Entwickler ist die Aufgabe gestellt, zwischen den Faktoren Performanz, Realismus und Umsetzbarkeit einen Vernünftigen Kompromiss zu finden. Die Unterschiedlichkeit dieser Kompromissfindung kann in~\ref{sec:hitbox} am Beispiel von Minecraft und CSGO betrachtet werden. Realismus ist dort kein ausschlaggebender Faktor für Erfolg.\\

Auch in diesem Projekt müssen die Erwartungen an physikalischen Realismus relativiert werden. Es bestehen Freiheitsgrade in sowohl der Definition der physikalischen Vorgänge als auch der algorithmischen Umsetzung. 
Videospiele, als Kunstform, können so auch surreale Konzepte umsetzen und Echzeitanforderungen können leichter eingehalten werden.


\section{Zusammenfassung}
Da die Arbeit an der Codebasis nicht auf das HSP-Projekt beschränkt ist, ist eine zukünftige Arbeit an einer Verbesserung der Synchronisation wahrscheinlich. Der aktuelle Stand weist noch einige Probleme und einiges an Optimierungspotenzial auf:
\begin{itemize}
\item Die Sichtrichtung einer Spielerfigur, und somit die grafisch angezeigte Perspektive am Client, wird aktuell noch vom Server bestimmt, ohne dass der Client für die Kamera im 3D-Raum eine eigene Schätzung verwendet. 
Ein Spieler muss so beim Drehen seiner Sicht die Zwei-Wege-Latenz zum Server in Kauf nehmen. Zwar handelt der Server nach dem hier gezeigten Synchronisationsschema mit Gleichzeitigkeit, der Unterschied wird jedoch am Client grafisch merkbar, wenn sich die Perspektive schnell ändert während z.B.~geschossen wird. Projektile sind prinzipiell da wo sie sein sollen, das lokal angezeigte Fadenkreuz u.U.~allerdings nicht. Für einen Shooter ist die Flüssigkeit des Bildes  beim Zielen essentiell.
Dies wäre somit die höchste Priorität für Erweiterungen des klassischen linearen Extrapolationsmodells der Grafik am Client, die lokal anliegenden Eingabeereignisse eigenständig mit einzubeziehen. Problematisch bei der Umsezung ist dort, dass der Server bestimmt, wann der Spieler seine Waffe abfeuert. Er bestimmt damit auch den Rückstoß, der die Blickrichtung ändert.
\item Geräusche, die alle als Reaktion auf ein Ereignis im Spiel auftreten (also vom Server angeordnet werden), wurden im Rahmen des Projekts nicht synchronisiert. Der Grund dafür ist, das aufgrund der Art und Weise, wie diese bisher implementiert sind und eine weitgehende Neuimplementierung des Features notwendig wäre, um auf dem Client zu funktionieren. Das wurde für den Rahmen des Projekts als zu Aufwändig eingeschätzt.
\item Informationen sind aktuell nicht priorisiert. Es wäre jedoch für die gefühlte Latenz von Vorteil, wenn Updates z.B.~zur eigenen Spielfigur in jedem Paket enthalten wären, und nicht nur, wenn diese im Round-Robin Verfahren an der Reihe is. Da aber aktuell der Syncable-Manager nur ein einziges Paket erzeugt, das von der Netzwerkkomponente an alle Clients gesendet wird, muss dort eine signifikante Anpassung erfolgen, um für unterschiedliche Clients unterschiedliche Updates zusammenzustellen. Über ein ähnliches System könnte man außerdem die Genauigkeit von Update-Paketen auf den konkreten Client zuschneiden. So wäre es z.B. vertretbar, bei einem weit entfernten Gegner dessen Sichtrichtung nur mit einer Genauigkeit von 1\,Byte pro Achse zu übertragen.
\item Die bisherige Implementierung erlaubt neben einzelnen Kreierungen und Löschungen von Objekten auch die Einbindung beliebiger Events. Über diese könnte man bisherige bandbreitenintensive Kreierungsbefehle zu Gruppen zusammenfassen, die z.B. mit einem deterministischen Pseudozufallsgenerator auf beiden Maschinen mit minimalem Datenaustausch identische Resultate liefern. Man könnte diese Events außerdem benutzen, um die Updates für eine große Menge an Entitäten zu ersetzen. Dies wäre möglich, indem diese in einer Gruppe zusammengefasst wären und intern deterministisch agieren (z.B Gegner marschieren in Formation). Die übertragenen Events wären dann Befehle, die deterministisch die Aktionen über einen bestimmten Zeitraum bestimmen.
\item Der Client könnte komplexere Vorausberechnungen als die lineare Extrapolation anstellen. So könnte er z.B. temporäre Entitäten erzeugen, die nur solange existieren, bis Serverinformationen über deren Erstellungszeitpunkt den Client erreicht haben. Sie würden dann durch die aktuellen Informationen vom Server ersetzt werden (oder ersatzlos gestrichen, wenn der Server für diesem Zeitpunkt keine Erzeugung einer Entität vorgesehen hat). So könnte der Spieler z.B. lokal ein abgeschossenes Projektil sofort sehen, nicht erst wenn es vom Server bestätigt wurde.
\end{itemize}

\newpage
\printbibliography


%\bibliographystyle{hieeetr}
%\addcontentsline{toc}{chapter}{Bibliography}
%\bibliography{grr.bib}
%\begin{thebibliography}{12}
%\bibitem[HaKT1 98]{HaKT1 98} \footnote{In die 
%Bibliographie sollte s"amtliche benutzte Literatur 
%rein, auch nicht beim eigenen Vortrag angegebene, aber benutzte Papiere 
%und B\"ucher. Gleichzeitig sollte aber alles in der Literaturliste angegebene
%mindestens einmal im Artikel zitiert werden, sonst nicht auflisten.}
        %Michael Harkavy, J. D. Tygar, Hiroaki Kikuchi: {\sl Multi-round 
        %Anonymous Auction Protocols}; 1st IEEE Workshop on Dependable and 
        %Real-Time E-Commerce Systems, 1998.

%%Here go sources, which both grr and hel might need
% for example
\bibitem[1]{cd2D}
	Jimenez-Delgado, Juan \& Segura, Rafael \& Feito, Francisco. (2004). Efficient Collision Detection between 2D Polygons.. Journal of WSCG. 12. 191-198. 



%






%\bibitem[TseDef\_80]{tse}
        Seitz Default: {\sl How to Cite something 2}; 
        Communications of the ACM 22/11 (1979), S. 612-613.

%\end{thebibliography}

\end{document}





